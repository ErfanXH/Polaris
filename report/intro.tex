\chapter{مقدمه} 
\section{بیان مسئله}

با گسترش روزافزون شبکه‌های تلفن همراه و تنوع خدمات ارائه‌شده بر بستر آن‌ها، ارزیابی دقیق کیفیت شبکه و تجربه‌ی کاربری به یک ضرورت کلیدی برای اپراتورها، نهادهای نظارتی و حتی کاربران عادی تبدیل شده است. چالش اصلی در این حوزه، نبود یک سامانه‌ی یکپارچه و دقیق برای جمع‌آوری، پردازش و تحلیل داده‌های میدانی از دیدگاه کاربر نهایی است.  

روش‌های سنتی ارزیابی شبکه معمولاً متکی بر ابزارهای تخصصی و آزمون‌های آزمایشگاهی هستند که هزینه‌بر بوده و لزوماً شرایط واقعی استفاده‌ی کاربران را منعکس نمی‌کنند. این موضوع باعث ایجاد شکاف میان شاخص‌های فنی ثبت‌شده توسط اپراتورها و تجربه‌ی واقعی کاربران می‌شود.  

بنابراین، نیاز به سیستمی که بتواند به‌صورت خودکار، مداوم و در شرایط واقعی داده‌های عملکرد شبکه را جمع‌آوری کرده، پردازش و تحلیل کند و نتایج را به‌صورت قابل فهم و عملیاتی ارائه دهد، بیش از پیش احساس می‌شود.


\section{تعریف پروژه}

پروژه‌ی \lr{Polaris} یک سامانه‌ی جامع برای پایش و تحلیل کیفیت شبکه‌های تلفن همراه است که با هدف ارائه‌ی داده‌های دقیق و قابل اتکا از تجربه‌ی واقعی کاربران طراحی و پیاده‌سازی شده است. این سامانه با ترکیب ابزارهای جمع‌آوری داده، پردازش متمرکز، و ارائه‌ی گزارش‌های تحلیلی، امکان ارزیابی مستمر وضعیت شبکه و شناسایی نقاط ضعف و قوت آن را فراهم می‌کند.  

در این پروژه، داده‌های میدانی به‌صورت خودکار از محیط واقعی جمع‌آوری شده، پردازش و تحلیل می‌شوند و نتایج به شکل قابل فهم و کاربردی در اختیار کاربران ذی‌نفع قرار می‌گیرد. این رویکرد شکاف میان شاخص‌های فنی اپراتور و تجربه‌ی واقعی کاربر را کاهش می‌دهد و بستری برای تصمیم‌گیری مبتنی بر داده فراهم می‌سازد.

\section{هدف کلی}

هدف کلی پروژه‌ی \lr{Polaris} طراحی و پیاده‌سازی سامانه‌ای یکپارچه برای پایش و تحلیل کیفیت شبکه‌های تلفن همراه از دیدگاه کاربر نهایی است. این سامانه با جمع‌آوری داده‌های میدانی در شرایط واقعی، پردازش هوشمندانه و ارائه نتایج به‌صورت گزارش‌ها و نمودارهای تحلیلی، امکان ارزیابی دقیق‌تر کیفیت خدمات شبکه را فراهم می‌سازد.

این پروژه می‌کوشد با ارائه ابزارهایی برای اندازه‌گیری پارامترهای مختلف شبکه مانند کیفیت سیگنال، سرعت انتقال داده، و زمان پاسخ‌گویی سرویس‌ها، شکاف موجود بین شاخص‌های فنی اپراتورها و تجربه واقعی کاربران را کاهش دهد و به بهبود کیفیت خدمات کمک نماید.

\section{اجزاء پروژه}

پروژه‌ی \lr{Polaris} از سه بخش اصلی تشکیل شده است که هر یک نقش مکملی در جمع‌آوری، پردازش و نمایش داده‌های کیفیت شبکه ایفا می‌کنند:

\begin{enumerate}
    \item \textbf{کلاینت اندرویدی:}  
    یک برنامه موبایل توسعه‌یافته با زبان \lr{Kotlin} که وظیفه‌ی انجام تست‌های مختلف شبکه از جمله اندازه‌گیری سرعت دانلود و آپلود، زمان پاسخ \lr{Ping}، زمان پاسخ \lr{DNS}، ارسال و دریافت پیامک، و اندازه‌گیری پارامترهای کیفیت سیگنال را بر عهده دارد. داده‌های حاصل به‌صورت ساختاریافته به سرور ارسال می‌شوند.

    \item \textbf{بک‌اند:}  
    سروری مبتنی بر فریم‌ورک \lr{Django} که وظیفه‌ی دریافت، اعتبارسنجی، پردازش و ذخیره‌سازی داده‌ها در پایگاه داده \lr{MySQL} را دارد. این بخش از طریق \lr{API}هایی امن، ارتباط میان کلاینت‌ها و رابط کاربری وب را مدیریت می‌کند.

    \item \textbf{رابط کاربری وب:}  
    یک پنل مدیریتی و تحلیلی مبتنی بر فناوری‌های وب که داده‌های ذخیره‌شده را در قالب نقشه‌های تعاملی، نمودارهای تحلیلی و جداول قابل جست‌وجو نمایش می‌دهد و امکان تحلیل و استخراج گزارش‌های مختلف را فراهم می‌کند.
\end{enumerate}
