\chapter{فرانت‌اند}
    \section{ساختار و جزئیات \lr{Web Application}}
    
    این بخش به بررسی ساختار و اجزای اصلی برنامه وب می‌پردازد. برنامه با استفاده از کتابخانه \lr{React} و ابزار \lr{Vite} توسعه داده شده و از معماری ماژولار برای جداسازی مسئولیت‌ها و بهبود نگهداشت‌پذیری استفاده می‌کند.
    
    \subsection{فایل‌ها و پیکربندی سطح بالا}
    در ریشه پروژه، تعدادی فایل پیکربندی و مستندات وجود دارد:
    \begin{itemize}
    	\item \textbf{\lr{package.json}} و \textbf{\lr{package-lock.json}}: مدیریت وابستگی‌ها و اسکریپت‌های پروژه.
    	\item \textbf{\lr{vite.config.js}}: تنظیمات مربوط به ابزار ساخت \lr{Vite}.
    	\item \textbf{\lr{eslint.config.js}}: پیکربندی استانداردهای کدنویسی و بررسی استاتیک کد.
    	\item \textbf{\lr{index.html}}: نقطه ورود HTML برنامه.
    	\item \textbf{\lr{README.md}}: توضیحات و راهنمای پروژه.
    	\item \textbf{\lr{Dockerfile}}: پیکربندی ساخت و اجرای برنامه در محیط \lr{Docker}.
    \end{itemize}
    
    \subsection{پوشه \lr{public}}
    این پوشه شامل منابع استاتیک مانند تصاویر و آیکون‌ها است که بدون پردازش توسط \lr{Vite} در دسترس خواهند بود:
    \begin{itemize}
    	\item \lr{logo.svg}، \lr{logo\_icon.svg}: لوگوهای برنامه.
    	\item \lr{phone.jpeg} و \lr{phone\_dark.jpeg}: تصاویر پیش‌نمایش برنامه در حالت روشن و تاریک.
    	\item \lr{content.svg}: عناصر گرافیکی برای بخش‌های مختلف.
    \end{itemize}
    
    \subsection{پوشه \lr{src}}
    پوشه \lr{src} محل اصلی کدهای برنامه است و شامل چند بخش مهم می‌شود:
    
    \subsubsection{\lr{App.jsx}، \lr{main.jsx}، \lr{router.jsx}}
    \begin{itemize}
    	\item \lr{App.jsx}: کامپوننت اصلی که ساختار کلی برنامه را تعریف می‌کند.
    	\item \lr{main.jsx}: نقطه شروع \lr{React} که برنامه را در عنصر اصلی HTML بارگذاری می‌کند.
    	\item \lr{router.jsx}: مسیرها و صفحات مختلف برنامه را با استفاده از کتابخانه ناوبری تعریف می‌کند.
    \end{itemize}
    
    \subsubsection{پوشه \lr{context}}
    این بخش شامل فایل \lr{Authorization.jsx} است که \lr{Context} مربوط به وضعیت احراز هویت و مجوزهای کاربر را فراهم می‌کند.
    
    \subsubsection{پوشه \lr{hooks}}
    هوک‌های سفارشی برای مدیریت داده‌ها و منطق برنامه:
    \begin{itemize}
    	\item \lr{useDashboardData.js}: مدیریت و دریافت داده‌های داشبورد.
    	\item \lr{useUserList.js}: مدیریت لیست کاربران و داده‌های مربوطه.
    \end{itemize}
    
    \subsubsection{پوشه \lr{managers}}
    این بخش وظایف مدیریتی و ارتباط با API را برعهده دارد:
    \begin{itemize}
    	\item \lr{ApiManager.js}: مدیریت درخواست‌های HTTP.
    	\item \lr{CookieManager.js}: مدیریت کوکی‌ها.
    	\item \lr{LoginManager.js}، \lr{SignUpManager.js}، \lr{ResetPasswordManager.js}،\\ \lr{VerifyManager.js}: مدیریت عملیات مربوط به احراز هویت.
    	\item \lr{DashboardManager.js}، \lr{MapManager.js}، \lr{UserListManager.js}: مدیریت داده‌های بخش‌های مختلف برنامه.
    	\item \lr{Constants.js}: تعریف مقادیر ثابت مورد استفاده.
    \end{itemize}
    
    \subsubsection{پوشه \lr{pages}}
    شامل صفحات مختلف رابط کاربری است. هر صفحه ممکن است پوشه \lr{components} مخصوص به خود را برای اجزای کوچکتر UI داشته باشد:
    \begin{itemize}
    	\item \lr{Dashboard}: نمایش داشبورد، نمودارها، فیلترها و جدول داده‌ها.
    	\item \lr{Landing}: صفحه اصلی معرفی برنامه شامل بخش‌هایی مانند ویژگی‌ها، دانلودها.
    	\item \lr{Login}، \lr{SignUp}، \lr{Verify}، \lr{ResetPassword}: صفحات مربوط به فرآیند ورود و ثبت‌نام.
    	\item \lr{Map}: نمایش نقشه و شاخص‌های سیگنال.
    	\item \lr{UserList}: نمایش اطلاعات کاربران برای ادمین.
    	\item \lr{NotFound}: صفحه خطای ۴۰۴.
    \end{itemize}
    
    \subsubsection{پوشه \lr{utils}}
    ابزارهای کمکی که در بخش‌های مختلف استفاده می‌شوند:
    \begin{itemize}
    	\item \lr{DatetimeUtility.js} و \lr{FormatDatetime.js}: قالب‌بندی و پردازش تاریخ و زمان.
    	\item \lr{MapUtils.js}: توابع کمکی برای کار با نقشه.
    	\item \lr{ThemeManager.js}: مدیریت تم و حالت روشن و تاریک برنامه.
    \end{itemize}
    \section{احراز هویت}
    در این بخش، به تشریح معماری و پیاده‌سازی سیستم احراز هویت در پروژه می‌پردازیم. این سیستم بر اساس ساختار ماژولار و تفکیک وظایف \lr{(separation of concerns)} طراحی شده است.
    \subsection{مدیریت وضعیت \lr{(State Management)}}
    \begin{itemize}
    	\item \lr{src/context/Authorization.jsx}: این فایل هسته‌ی اصلی مدیریت وضعیت احراز هویت در کل برنامه است و با استفاده از \lr{React Context API} پیاده‌سازی شده است. وظایف اصلی آن شامل موارد زیر است:
    	\begin{itemize}
    		\item نگهداری وضعیت ورود کاربر (\lr{isAuthenticated})، وضعیت مدیر بودن (\lr{isAdmin}) و وضعیت بارگذاری (\lr{isLoading}).
    		\item مقداردهی اولیه‌ی وضعیت با استفاده از داده‌های ذخیره‌شده در کوکی‌ها توسط \lr{CookieManager}.
    		\item ارائه‌ی متد \lr{setAuthentication} برای فعال‌کردن وضعیت احراز هویت پس از ورود موفق کاربر.
    		\item ارائه‌ی متد \lr{resetAuthentication} برای بازنشانی وضعیت احراز هویت در هنگام خروج یا منقضی‌شدن توکن.
    		\item استفاده از \lr{useCallback} برای جلوگیری از ایجاد توابع جدید در هر بار رندر و بهبود کارایی.
    	\end{itemize}
    \end{itemize}
    
    \subsection{مسیریابی محافظت‌شده (\lr{Protected Routing})}
    \begin{itemize}
    	\item در فایل \lr{src/context/Authorization.jsx} همچنین یک کامپوننت به نام \lr{ProtectedRoute} پیاده‌سازی شده است که وظیفه‌ی کنترل دسترسی به مسیرهای حساس را بر عهده دارد. این کامپوننت:
    	\begin{itemize}
    		\item در صورت فعال بودن حالت \lr{isLoading}، یک پیام بارگذاری به کاربر نمایش می‌دهد.
    		\item اگر مسیر \lr{adminOnly} باشد، تنها زمانی اجازه دسترسی می‌دهد که کاربر هم مدیر باشد و هم احراز هویت شده باشد.
    		\item در سایر مسیرهای محافظت‌شده، تنها کافی است کاربر احراز هویت شده باشد.
    		\item در صورت عدم احراز هویت، کاربر را به مسیر ورود (\lr{/login}) هدایت می‌کند.
    	\end{itemize}
    \end{itemize}
    
    \subsection{راه‌اندازی سراسری (\lr{Global Setup})}
    \begin{itemize}
    	\item \lr{src/App.jsx}: این فایل نقطه‌ی ورود اصلی رابط کاربری است که در آن ارائه‌دهنده‌های سراسری (Global Providers) مقداردهی می‌شوند:
    	\begin{itemize}
    		\item \lr{ThemeProvider} برای مدیریت تم روشن و تاریک با استفاده از ترجیحات سیستم.
    		\item \lr{ToastContainer} برای نمایش اعلان‌ها به کاربر.
    		\item \lr{QueryClientProvider} برای مدیریت درخواست‌های داده با \lr{@tanstack/react-query}.
    		\item \lr{AuthProvider} که از \lr{Authorization.jsx} وارد شده و مسئولیت ارائه‌ی وضعیت احراز هویت به تمام بخش‌های برنامه را دارد.
    		\item \lr{RouterProvider} برای بارگذاری مسیرها طبق پیکربندی \lr{router.jsx}.
    	\end{itemize}
    \end{itemize}
    
    \subsection{مدیریت منطق تجاری (\lr{Business Logic})}
    \begin{itemize}
    	\item \textbf{src/managers/}: این دایرکتوری شامل ماژول‌هایی است که منطق تجاری احراز هویت و ارتباط با API را از رابط کاربری جدا می‌کنند تا کد ساختارمند و قابل نگهداری باشد.
    	\begin{itemize}
    		\item \textbf{ApiManager.js}: یک لایه انتزاعی برای برقراری ارتباط با API که با استفاده از کتابخانه \lr{axios} پیاده‌سازی شده است. این ماژول توکن احراز هویت را از \lr{CookieManager} بارگذاری کرده و به هدر درخواست‌ها اضافه می‌کند. همچنین در صورت دریافت خطای 401، کاربر را به صفحه ورود هدایت می‌کند.
    		\item \textbf{Constants.js}: شامل ثابت‌های سراسری مانند آدرس پایه API، دامنه اصلی و کلید ذخیره‌سازی کوکی است تا وابستگی به مسیرها و مقادیر ثابت متمرکز و قابل تغییر باشد.
    		\item \textbf{CookieManager.js}: مسئول ذخیره‌سازی، بازیابی و حذف امن داده‌های احراز هویت (توکن و وضعیت مدیر بودن) در کوکی‌های مرورگر است. این ماژول از کتابخانه \lr{js-cookie} برای مدیریت کوکی‌ها استفاده می‌کند.
    		\item \textbf{LoginManager.js}: منطق ورود کاربر را پیاده‌سازی کرده و درخواست را به نقطه پایانی مشخص ارسال می‌کند. در صورت ورود موفق، اطلاعات دریافتی را در کوکی ذخیره کرده و خطاهای احتمالی (مانند عدم تأیید ایمیل) را مدیریت می‌کند.
    		\item \textbf{SignUpManager.js}: منطق ثبت‌نام کاربر جدید را از طریق ارسال درخواست به نقطه پایانی معین مدیریت می‌کند. خطاهای برگشتی از API را پردازش کرده و به لایه رابط کاربری ارسال می‌کند.
    		\item \textbf{ResetPasswordManager.js}: مسئول مدیریت فرآیند بازیابی رمز عبور شامل سه مرحله است: ارسال کد تأیید، بررسی صحت کد و تنظیم رمز عبور جدید. تمامی این مراحل با نقاط پایانی مشخص API انجام می‌شود.
    		\item \textbf{VerifyManager.js}: فرآیند تأیید هویت کاربر را با ارسال کد تأیید و اطلاعات کاربری به API انجام می‌دهد. در صورت موفقیت، داده‌های کاربر در کوکی ذخیره می‌شوند تا وضعیت احراز هویت در سراسر برنامه فعال شود.
    	\end{itemize}
    \end{itemize}
    
    \subsection{لایه رابط کاربری (\lr{UI Layer})}
    	\begin{itemize}
    		\item \textbf{src/pages/}: این دایرکتوری شامل کامپوننت‌های صفحه‌ای است که با کاربر تعامل مستقیم دارند و داده‌ها را از کاربر دریافت یا به او نمایش می‌دهند.
    		\begin{itemize}
    			\item \textbf{Login/index.jsx}: این کامپوننت صفحه ورود کاربر را پیاده‌سازی می‌کند و شامل یک فرم با اعتبارسنجی سمت کاربر با استفاده از کتابخانه‌های \lr{react-hook-form} و \lr{zod} است. فیلد \lr{number\_or\_email} بررسی می‌کند که ورودی یک ایمیل معتبر یا شماره تلفن با فرمت صحیح باشد و رمز عبور حداقل ۸ کاراکتر باشد.  
    			پس از ارسال فرم، داده‌ها به متد \lr{login} از \lr{LoginManager.js} ارسال می‌شوند تا فرآیند احراز هویت سمت سرور انجام شود. در صورت موفقیت، پیام موفقیت با استفاده از \lr{react-toastify} نمایش داده شده و کاربر به داشبورد هدایت می‌شود.  
    			در صورت بروز خطای 401، کاربر به صفحه تأیید حساب (\lr{/verify}) منتقل می‌شود و اطلاعات ورودش همراه کد تأیید ارسال می‌شود.  
    			این صفحه از \lr{Material-UI} برای طراحی رابط کاربری واکنش‌گرا استفاده می‌کند و قابلیت نمایش/مخفی‌سازی رمز عبور را فراهم کرده است.  
    			همچنین لینک‌های ناوبری برای بازیابی رمز عبور (\lr{/reset-password}) و ثبت‌نام (\lr{/sign-up}) در صفحه قرار داده شده‌اند تا تجربه کاربری کامل‌تری ارائه شود.
    			\item \textbf{SignUp/index.jsx}: این کامپوننت مسئول نمایش و مدیریت فرم ثبت‌نام کاربر است. 
    			از کتابخانه \lr{react-hook-form} همراه با \lr{zod} برای اعتبارسنجی داده‌ها استفاده می‌کند. 
    			اسکیما \lr{signUpSchema} شامل اعتبارسنجی ایمیل، شماره تلفن، رمز عبور و تأیید رمز عبور است، و از ویژگی \lr{refine} برای اطمینان از تطابق رمز عبور و تکرار آن بهره می‌برد. 
    			برای مدیریت وضعیت نمایش/عدم نمایش رمز عبور و تأیید آن، دو state محلی تعریف شده است. 
    			در تابع \lr{onSubmit}، داده‌های معتبر از طریق \lr{SignUpManager.signUp} به سرور ارسال شده و در صورت موفقیت، کاربر به صفحه تأیید هدایت می‌شود و پیام موفقیت با \lr{react-toastify} نمایش داده می‌شود. 
    			در صورت خطا، پیام مناسب خطا به کاربر نشان داده می‌شود. 
    			این فرم با استفاده از کامپوننت‌های \lr{Material-UI} طراحی شده و شامل ورودی‌های شماره تلفن، ایمیل، رمز عبور، تأیید رمز عبور، و لینک تغییر مسیر به صفحه ورود است. 
    			نمایش بصری حالت بارگذاری نیز در دکمه ثبت‌نام پیاده‌سازی شده است.
    			
    			\item \textbf{ResetPassword/index.jsx}: کامپوننت مربوط به فرآیند بازیابی رمز عبور که در سه مرحله پیاده‌سازی شده است.  
    			\begin{enumerate}
    				\item \textbf{مرحله ۱ (دریافت ایمیل یا شماره موبایل)}: با استفاده از \lr{react-hook-form} و \lr{zod} ورودی کاربر اعتبارسنجی شده و با متد \lr{ResetPasswordManager.sendResetCode} کد تأیید ارسال می‌شود.  
    				\item \textbf{مرحله ۲ (تأیید کد)}: کاربر یک کد ۵ رقمی را وارد کرده و این کد با متد \\ \lr{ResetPasswordManager.verifyResetCode} بررسی می‌شود. امکان ارسال مجدد کد با محدودیت زمانی نیز فراهم شده است.  
    				\item \textbf{مرحله ۳ (ایجاد رمز عبور جدید)}: کاربر رمز عبور جدید را وارد کرده و تکرار آن نیز بررسی می‌شود. اعتبارسنجی با \lr{zod} انجام و داده‌ها با متد \\\lr{ResetPasswordManager.resetPassword} به سرور ارسال می‌شوند.  
    				\item \textbf{مدیریت وضعیت}: با استفاده از \lr{useState} و \lr{useForm} وضعیت‌هایی مانند مرحله فعلی، بارگذاری (\lr{loading})، شمارش معکوس و نمایش/مخفی‌کردن رمز مدیریت می‌شود.  
    				\item \textbf{رابط کاربری}: با کتابخانه \lr{Material-UI} طراحی شده و از کامپوننت‌هایی مانند \\ \lr{TextField}، \lr{Button} و \lr{CircularProgress} استفاده شده است.  
    				\item \textbf{بازخورد به کاربر}: پیام‌های موفقیت یا خطا با کتابخانه \lr{react-toastify} نمایش داده می‌شوند.  
    				\item \textbf{ناوبری}: پس از موفقیت در تغییر رمز عبور، کاربر به صفحه ورود هدایت می‌شود. برای جابه‌جایی بین مراحل نیز از \lr{useNavigate} استفاده شده است.  
    			\end{enumerate}
    			
    			\item \textbf{Verify/index.jsx}: صفحه‌ای برای تأیید هویت کاربر که پس از ثبت‌نام یا ورود نیازمند تأیید ایمیل یا شماره است. این صفحه با استفاده از کتابخانه‌های \lr{React Hook Form} و \lr{Zod} اعتبارسنجی کد پنج‌رقمی را انجام می‌دهد. داده‌های کاربر مانند ایمیل و رمز عبور از \lr{location.state} دریافت می‌شوند. تابع \lr{onSubmit} درخواست تأیید را به \lr{VerifyManager} ارسال کرده و در صورت موفقیت، کاربر را وارد سیستم می‌کند. امکان ارسال مجدد کد با شمارش معکوس نیز پیاده‌سازی شده است تا از ارسال مکرر جلوگیری شود. طراحی و استایل صفحه با \lr{Material UI} و تم پروژه هماهنگ شده و از \lr{CircularProgress} برای نمایش وضعیت بارگذاری استفاده می‌شود. همچنین از \lr{Toastify} برای نمایش پیام‌های موفقیت یا خطا بهره گرفته شده است. لینک بازگشت به صفحه ورود نیز در انتها قرار داده شده تا کاربر بتواند مسیر خود را تغییر دهد.
    			
    		\end{itemize}
    	\end{itemize}
    
    \section{داشبورد}
    
    \subsection{مدیریت منطق تجاری (\lr{Business Logic})}
    \begin{itemize}
    	\item \textbf{managers/DashboardManager.js}: این ماژول وظیفه مدیریت منطق مربوط به دریافت داده‌های داشبورد را بر عهده دارد.  
    	در این فایل، درخواست‌های لازم برای دریافت اطلاعات اندازه‌گیری‌های موبایل به نقطه پایانی API ارسال می‌شود.  
    	در صورت موفقیت، داده‌ها به صورت مستقیم برگردانده می‌شوند و در صورت خطا، پیام خطا به صورت دقیق پردازش و بازگردانده می‌شود.  
    	این ماژول لایه‌ای میان رابط کاربری و API است که به جداسازی مسئولیت‌ها کمک کرده و مدیریت خطاها را به صورت متمرکز انجام می‌دهد.  
    	به طور خلاصه، این فایل یک نقطه ورود ساده و کارا برای واکشی داده‌های خام داشبورد فراهم می‌کند.  
    \end{itemize}
    
    \subsection{هوک‌های سفارشی (\lr{Custom Hooks})}
    \begin{itemize}
    	\item \textbf{hooks/useDashboardData.js}: یک هوک سفارشی در \lr{React} است که مدیریت کامل داده‌ها و حالت‌های مرتبط با داشبورد را بر عهده دارد.  
    	این هوک ابتدا با استفاده از \lr{React Query}، داده‌های اندازه‌گیری را از \lr{DashboardManager} واکشی می‌کند و وضعیت بارگذاری و خطا را کنترل می‌نماید.  
    	سپس داده‌ها را بر اساس فیلترهای کاربر مانند نوع شبکه و بازه زمانی به صورت پویا فیلتر می‌کند.  
    	علاوه بر این، این هوک مجموعه‌ای از گزینه‌های پیکربندی شده برای انواع نمودارهای مختلف را با توجه به داده‌های فیلتر شده و تم رابط کاربری تولید می‌کند.  
    	استفاده از این هوک باعث جداسازی کامل منطق داده و آماده‌سازی آن برای نمایش در کامپوننت‌ها شده و کد رابط کاربری را ساده‌تر و قابل نگهداری‌تر می‌کند.  
    	همچنین بازخوانی داده‌ها به صورت خودکار و با فاصله زمانی مشخص انجام می‌شود تا داشبورد همواره به‌روز باشد.  
    	در نهایت، این هوک توابع و متغیرهای مورد نیاز برای فیلترینگ و تغییر حالت داده‌ها را نیز ارائه می‌دهد که به راحتی در کامپوننت‌های مختلف قابل استفاده است.
    \end{itemize}
    
    \subsection{لایه رابط کاربری (\lr{UI Layer})}
    \begin{itemize}
    	\item \textbf{pages/UserLayout/index.jsx}: این کامپوننت اصلی لایه کاربری (User Layout) برنامه را پیاده‌سازی می‌کند و ساختار کلی صفحات را با یک نوار کناری (Sidebar) و نوار بالایی (AppBar) فراهم می‌آورد.  
    	این لایه به صورت واکنش‌گرا طراحی شده و در دستگاه‌های موبایل، منوی کشویی بازشو دارد، در حالی که در صفحه‌نمایش‌های بزرگ‌تر یک منوی ثابت با قابلیت جمع‌شدن (Collapse) نمایش داده می‌شود.  
    	منوی کناری شامل آیتم‌های ناوبری مانند داشبورد، نقشه، لیست کاربران (در صورت مدیر بودن کاربر) و گزینه خروج است.  
    	کاربر می‌تواند با کلیک روی هر آیتم به صفحه مربوطه هدایت شود و گزینه خروج، کوکی‌های احراز هویت را حذف کرده و کاربر را به صفحه ورود می‌برد.  
    	طراحی این کامپوننت با استفاده از کتابخانه \lr{Material-UI} انجام شده و از امکاناتی مانند \lr{Drawer}، \lr{AppBar}، آیکون‌ها و واکنش‌پذیری بهره می‌برد.  
    	همچنین وضعیت باز یا بسته بودن منو و حالت جمع‌شده در \lr{state} محلی نگهداری می‌شود و با تغییر اندازه صفحه نمایش رفتار منو نیز تغییر می‌کند.  
    	ناوبری از طریق هوک \lr{useNavigate} انجام می‌شود و لایه احراز هویت با استفاده از \lr{context} مدیریت می‌گردد.  
    	این کامپوننت همچنین فضای لازم برای بارگذاری صفحات فرزند را با \lr{Outlet} فراهم می‌کند تا بتوان صفحات مختلف را به صورت درون‌خطی نمایش داد.  
    	به طور کلی، این فایل مسئول ایجاد ساختار پایه‌ای و یکپارچه برای صفحات مختلف کاربری با قابلیت ناوبری و مدیریت دسترسی است.
    	
    	\item \textbf{pages/Dashboard/index.jsx}:  
    	این کامپوننت صفحه اصلی داشبورد شبکه را پیاده‌سازی می‌کند و شامل نمایش داده‌ها، فیلترها و نمودارهای متنوع است.  
    	با استفاده از هوک سفارشی\\ \lr{useDashboardData} داده‌ها واکشی و فیلتر شده و وضعیت بارگذاری کنترل می‌شود.  
    	داشبورد شامل چندین تب با عناوین مختلف است که هر تب داده‌ها و نمودارهای خاص خود را نمایش می‌دهد، مانند \lr{Overview}، \lr{Measurements}، \lr{SMS}، \lr{Web} و سایر موارد.  
    	در تب Overview نمودارهای کلی تکنولوژی شبکه، توزیع \lr{ARFCN}، باند فرکانسی، پراکندگی \lr{RSRP} و \lr{RSRQ} و جدول داده‌ها ارائه می‌شود.  
    	در تب‌های تخصصی‌تر مانند \lr{SMS} یا \lr{Web}، نمودارهای خطی، جعبه‌ای (\lr{BoxPlot}) و توزیع داده‌ها همراه با جدول آمارهای عددی نمایش داده می‌شوند.  
    	این کامپوننت برای نمایش واکنش‌گرا طراحی شده و در دستگاه‌های موبایل تنظیمات فاصله و چینش متفاوت دارد.  
    	مدیریت وضعیت تب فعال و رندر نمودارها با استفاده از \lr{state} و \lr{conditional rendering} انجام می‌شود.  
    	علاوه بر نمودارها، جدول داده‌ها با قابلیت پاسخگویی به تغییر تب‌ها، به‌روزرسانی می‌شود تا داده‌ها را به صورت جزئی‌تر نمایش دهد.  
    	این ساختار باعث می‌شود داشبورد به صورت جامع و کاربرپسند اطلاعات شبکه را ارائه دهد و تحلیل‌های متنوعی را امکان‌پذیر سازد.
    	
    	\item \textbf{pages/Dashboard/components/ColumnConfig.js}:  
    	این فایل پیکربندی ستون‌های جدول داشبورد را تعریف می‌کند.  
    	هر ستون شامل برچسب (label)، تابع رندر برای نمایش داده‌ها و سبک ظاهری اختصاصی است.  
    	برای مثال ستون زمان (timestamp) با استفاده از تابع \lr{formatDateTime} فرمت شده و به صورت خوانا نمایش داده می‌شود.  
    	سایر ستون‌ها مانند موقعیت جغرافیایی، نوع شبکه، باند فرکانسی، ARFCN و اطلاعات سلولی نیز به صورت مناسب فرمت و رندر می‌شوند.  
    	این ساختار به جداسازی منطق نمایش داده‌ها در جدول کمک کرده و قابلیت گسترش و نگهداری را بالا می‌برد.  
    	مقادیر نال یا نامشخص با استفاده از "-" نمایش داده می‌شوند تا جدول مرتب و قابل فهم باقی بماند.  
    	ستون‌هایی مانند سرعت دانلود و آپلود، پینگ و پاسخ DNS با دقت عددی مشخص شده نمایش داده می‌شوند.  
    	به طور کلی، این فایل قالب‌بندی و نمایش داده‌های خام را برای جدول داشبورد استاندارد می‌کند.  
    	
    	\item \textbf{pages/Dashboard/components/DashboardCharts.jsx}:  
    	این فایل مجموعه‌ای از کامپوننت‌های نمودار را برای داشبورد تعریف می‌کند که با استفاده از \lr{echarts-for-react} پیاده‌سازی شده‌اند.  
    	هر نمودار درون یک \lr{ChartContainer} قرار گرفته که قابلیت نمایش تمام صفحه (Fullscreen) را با استفاده از \lr{Dialog} فراهم می‌کند.  
    	کامپوننت‌های مختلفی مانند نمودار تکنولوژی شبکه، توزیع \lr{ARFCN}، نمودار باند فرکانسی، پراکندگی RSRP/RSRQ و نمودارهای سیگنال به مرور زمان در این فایل موجود است.  
    	این کامپوننت‌ها به صورت واکنش‌گرا طراحی شده و ارتفاع آن‌ها در موبایل و دسکتاپ متفاوت تنظیم می‌شود.  
    	استفاده از \lr{useTheme} باعث می‌شود تم نمودارها با حالت کلی رابط کاربری هماهنگ باشد (حالت تاریک یا روشن).  
    	آیکن \lr{Fullscreen} برای هر نمودار اجازه می‌دهد تا نمودارها به صورت بزرگ و واضح‌تر مشاهده شوند.  
    	این ساختار باعث یکپارچگی در نمایش نمودارها و تجربه کاربری بهتر در داشبورد می‌شود.  
    	همچنین کامپوننت‌های نمودار مشابه، با رابط یکسان قابل استفاده مجدد هستند و کد را تمیز و منظم نگه می‌دارند.
    	
    	\item \textbf{pages/Dashboard/components/DashboardConfig.js}:  
    	این فایل شامل توابعی برای ساخت گزینه‌ها (option) و تنظیمات نمودارهای مختلف داشبورد است که با کتابخانه ECharts ساخته می‌شوند.  
    	برای هر نوع نمودار مانند نمودار دایره‌ای شبکه‌های مخابراتی (\lr{Network Tech})، توزیع \lr{ARFCN}، نمودار میله‌ای باند فرکانسی و نمودار پراکندگی \lr{RSRP} و \lr{RSRQ}، تنظیمات مربوط تعریف شده است.  
    	توابع داده‌های ورودی را پردازش کرده و شمارش یا فیلترهای لازم را انجام می‌دهند تا داده‌ها به فرمت مناسب نمودار تبدیل شوند.  
    	مثلاً در نمودار دایره‌ای نوع شبکه، مواردی که مقدارشان "UNKNOWN" است نادیده گرفته می‌شوند.  
    	نمودارهای خطی مانند قدرت سیگنال در طول زمان، تعداد اندازه‌گیری‌ها بر اساس ساعت، و نمودارهای باکس‌پلات و توزیع داده‌ها نیز با محاسبات آماری دقیق آماده می‌شوند.  
    	تابع \lr{percentile} برای محاسبه درصدک‌ها در نمودار باکس‌پلات استفاده می‌شود تا بازه‌های آماری مانند چارک‌ها و رنج بین چارکی محاسبه شود.  
    	تنظیمات محورهای \lr{X} و \lr{Y}، عنوان‌ها، رنگ‌بندی‌ها و استایل‌ها با توجه به تم کلی برنامه و پارامترهای ورودی تنظیم می‌شوند.  
    	این فایل با جداسازی منطق ساخت گزینه‌های نمودار، به مدیریت بهتر و قابلیت نگهداری کد کمک می‌کند.  
    	همچنین انعطاف‌پذیری برای اضافه کردن نمودارهای جدید یا تغییر ظاهر آن‌ها را ساده‌تر می‌سازد.  
    	به طور کلی، این کامپوننت هستهٔ تولید تنظیمات نمودارهای داشبورد است که باعث یکپارچگی و سازگاری نمایشی می‌شود.
    	
    	\item \textbf{pages/Dashboard/components/DashboardFilters.jsx}:  
    	این کامپوننت مسئول نمایش فیلترهای داشبورد است که به کاربر امکان انتخاب نوع شبکه و بازه زمانی داده‌ها را می‌دهد.  
    	از \lr{TextField}های MUI برای انتخاب نوع شبکه و تاریخ شروع و پایان استفاده شده است که در حالت موبایل به صورت واکنش‌گرا تنظیم می‌شوند.  
    	لیست انواع شبکه شامل گزینه‌های مختلف مانند \lr{GSM}، \lr{LTE} و غیره است که کاربر می‌تواند به دلخواه فیلتر کند.  
    	برای فیلدهای تاریخ، استایل‌های سفارشی اعمال شده تا ظاهر دکمه انتخاب تاریخ (\lr{calendar picker}) در حالت تاریک و روشن متناسب باشد.  
    	تغییرات در فیلدهای تاریخ با استفاده از تابع کمکی \lr{LocalizeDateTime} به فرمت محلی تبدیل و در \lr{state} والد ذخیره می‌شوند.  
    	کد به صورت واکنش‌گرا نوشته شده و بسته به اندازه صفحه، اندازه فیلدها و چینش آن‌ها تغییر می‌کند.  
    	این کامپوننت بدون حالت داخلی (stateless) است و تمام داده‌ها و مدیریت حالت از والد به صورت props دریافت می‌شود.  
    	استفاده از \lr{flexWrap} و فاصله‌گذاری (gap) بین فیلدها باعث می‌شود چینش در موبایل و دسکتاپ مرتب و کاربرپسند باشد.  
    	این فیلترها نقش مهمی در محدود کردن داده‌های نمایش داده شده در نمودارها و جداول داشبورد دارند.  
    	در نهایت، این کامپوننت رابط کاربری ساده، تمیز و قابل گسترش برای فیلترکردن داده‌ها فراهم می‌کند.
    	
    	\item \textbf{pages/Dashboard/components/DashboardTable.jsx}:
    	این کامپوننت جدول داده‌های اندازه‌گیری شده را نمایش می‌دهد و از قابلیت صفحه‌بندی (pagination) پشتیبانی می‌کند.
    	ستون‌های جدول بر اساس تب فعال انتخاب می‌شوند و تعداد ردیف‌های نمایش داده شده در هر صفحه در موبایل و دسکتاپ متفاوت است.
    	امکان تغییر صفحه و تعداد ردیف‌ها توسط کاربر فراهم شده و به‌روزرسانی داده‌ها با توجه به این تغییرات انجام می‌شود.
    	دو دکمه برای صادر کردن داده‌ها به فرمت CSV و KML وجود دارد؛ CSV برای داده‌های جدولی و KML برای داده‌های جغرافیایی با مختصات مناسب.
    	در فرآیند صادرات \lr{KML}، فقط داده‌هایی که مختصات معتبر دارند در فایل قرار می‌گیرند و توضیحات مربوط به هر ردیف به صورت HTML داخل فایل درج می‌شود.
    	از توابع کمکی برای قالب‌بندی داده‌ها و نمایش سفارشی هر ستون استفاده شده است.
    	نمایش جدول با استفاده از کامپوننت‌های MUI انجام می‌شود تا تجربه کاربری بهتری در دستگاه‌های مختلف فراهم شود.
    	امکان پیمایش افقی جدول نیز برای نمایش بهتر در صفحه‌های کوچک وجود دارد.
    	این کامپوننت نقش مهمی در مدیریت، مشاهده و استخراج داده‌های اندازه‌گیری شده در داشبورد ایفا می‌کند.
    	تمام منطق مرتبط با نمایش داده‌ها، صفحه‌بندی و صادرات در این کامپوننت به صورت کامل پیاده‌سازی شده است.
    	
    	
    	\item \textbf{pages/Dashboard/components/TabConfig.js}:
    	این فایل تنظیمات ستون‌های نمایش داده شده در هر تب داشبورد را مشخص می‌کند.
    	هر تب شامل لیستی از کلیدهای ستون‌ها است که تعیین می‌کند کدام داده‌ها در آن تب نمایش داده شوند.
    	برای تب Overview همه ستون‌ها و برای تب‌های دیگر ستون‌های مرتبط با نوع داده خاص انتخاب شده‌اند.
    	
    	
    	\item \textbf{pages/Dashboard/components/ValueStatisticTable.jsx}:
    	این کامپوننت یک جدول آماری برای آرایه‌ای از مقادیر عددی نمایش می‌دهد.
    	با استفاده از کتابخانه \lr{simple-statistics}، شاخص‌هایی مانند میانگین، میانه، واریانس و انحراف معیار محاسبه می‌شود.
    	همچنین بازه‌های اطمینان \lr{68.2}٪، \lr{95}٪ و \lr{99.7}٪ بر اساس انحراف معیار ارائه می‌شود.
    	جدول نتیجه به شکل مرتب و با قالب‌بندی دو رقم اعشار نمایش داده می‌شود.
    	در صورت نبود داده، کامپوننت چیزی رندر نمی‌کند.
    	این کامپوننت برای نمایش خلاصه آماری داده‌ها و تحلیل سریع کاربرد دارد.
    	
    	
    \end{itemize}
    
    \subsection{ابزارهای کمکی (\lr{Utility Functions})}
    \begin{itemize}
    	\item \textbf{utils/DatetimeUtility.js}:
    	این فایل شامل دو تابع است:
    	تابع formatDateTime برای قالب‌بندی تاریخ و زمان به صورت رشته‌ای با فرمت \lr{YYYY-MM-DD HH:mm:ss}.
    	تابع LocalizeDateTime برای تبدیل تاریخ به زمان محلی با تنظیم اختلاف زمانی مشخص (کم کردن ۲۱۰ دقیقه).
    	
    	\item \textbf{utils/FormatDatetime.js}:
    	این فایل تنها شامل تابع formatDateTime است که همانند نسخه‌ی قبلی، تاریخ و زمان را به فرمت \lr{YYYY-MM-DD HH:mm:ss} تبدیل می‌کند.
    	در این نسخه، تابع برای قالب‌بندی تاریخ بدون تغییر منطقه زمانی استفاده می‌شود.
    	
    	
    \end{itemize}
    
\section{نقشه}
\label{sec:map}

\subsection{مرور کلی}
بخش \lr{Map} مسئول نمایش اندازه‌گیری‌های میدانی (\lr{Measurement}) بر روی یک نقشهٔ تعاملی است. این بخش از کتابخانهٔ \lr{react-leaflet} برای رندرِ نقشه استفاده می‌کند. منطقِ واکشی و عملیاتِ مرتبط با \lr{API} در یک ماژولِ مجزا (\lr{MapManager}) قرار دارد و اجزاء نمایش‌دهنده، پیکربندی‌ها و توابعِ کمکی در پوشهٔ \lr{pages/Map} و \lr{utils} پیاده‌سازی شده‌اند. هدفِ اصلی این بخش ارائهٔ نمایشِ بصریِ داده‌های اندازه‌گیری با امکان تنظیمِ پویا توسط کاربر و حفظِ پاسخ‌دهی مناسب است.

\subsection{ساختار کلی و نقش هر بخش}
مواردِ اصلی و نقشِ هر یک:
\begin{itemize}
  \item \textbf{\lr{MapManager}}: لایهٔ \lr{API} است که مسئولِ همهٔ تعامل‌ها با سرور شامل واکشیِ داده‌ها، فیلترِ سمت سرور (در صورت وجود پارامترها) و عملیاتِ حذف می‌باشد. \lr{MapManager} نقطهٔ متمرکزی برای مدیریت خطاها، افزودنِ هدرِ احراز هویت و یکسان‌سازی پاسخ‌ها فراهم می‌آورد؛ بنابراین بقیهٔ اجزاء نیازی به دانستن جزئیات پروتکلِ شبکه ندارند و تنها با دادهٔ نرمال‌شده کار می‌کنند.
  \item صفحهٔ \textbf{\lr{Map}}: صفحهٔ سطحِ بالا که مسئولِ ترکیبِ اجزاء است؛ ایجادِ \lr{MapContainer}، نگهداریِ \lr{state}هایِ محلی (مثلِ متریک انتخاب‌شده، پیکربندی رنگ‌ها و آستانه‌ها)، نمایشِ دیالوگِ تنظیمات و مدیریتِ \lr{legend} بر عهدهٔ این صفحه است. این صفحه همچنین نقش هماهنگ‌کننده را دارد: داده‌ها را از \lr{MapManager} می‌گیرد، آن‌ها را به شکل موردنیاز برای نمایش آماده می‌کند و به نشانگرها تحویل می‌دهد.
  \item \textbf{\lr{SignalStrengthMarker}}: جزئیاتِ نمایشِ هر \lr{Measurement} را بر عهده دارد — تعیینِ رنگ و آیکونِ مناسب، نمایشِ \lr{Popup} با اطلاعات تکمیلی و پاسخ به تعاملاتِ کاربر (کلیک، hover و ...) در این کامپوننت انجام می‌شود. این جز بهتر است تنها مسئولِ نمایش باشد و منطقِ پیچیدهٔ تبدیلِ داده را از لایه‌های دیگر دریافت کند.
  \item \textbf{\lr{default\_config}}: مجموعهٔ پیش‌فرضِ تنظیمات شامل دسته‌بندیِ متریک‌ها، واحدها، آستانه‌های اولیه و طیفِ رنگیِ پیشنهادی. این فایل مرجعِ اولیه برای هر تنظیم است و صفحهٔ \lr{Map} مقادیرِ فعلی را بر اساسِ آن مقداردهی می‌کند.
  \item \textbf{\lr{MapUtils}}: توابعِ کمکیِ غیرمستقیم (مثل تعیینِ بازهٔ مجازِ اسلایدرها، اندازهٔ گام تغییرات و محاسبهٔ رنگ متنِ متضاد) که به یک نمایشِ سازگار و قابل‌پیش‌بینی کمک می‌کنند.
\end{itemize}

\subsection{نحوهٔ واکشی و مدیریت داده‌ها}
واکنش‌گرِ اصلیِ واکشی معمولاً از \lr{react-query} و هوکِ \lr{useQuery} استفاده می‌کند تا:
\begin{itemize}
  \item کشِ مناسب و آستانهٔ تازه‌سازیِ خودکار فراهم شود،
  \item خطاها و حالت‌های بارگذاری به صورت متمرکز مدیریت شوند،
  \item سیاست‌هایی مثل refetch on window focus یا polling در صورت نیاز فعال یا غیرفعال شوند.
\end{itemize}
در سطحِ منطقی، جریانِ داده چنین است: صفحهٔ \lr{Map} از \lr{MapManager.getMeasurements(allData)} برای گرفتنِ لیستِ \lr{Measurement}ها استفاده می‌کند، پاسخ را بررسی و نرمال‌سازی می‌نماید و سپس به کامپوننت‌های نمایش‌دهنده پاس می‌دهد. \lr{MapManager} مسئولِ افزودنِ هدرِ احراز هویت (مثلاً از طریق \lr{CookieManager} یا \lr{ApiManager})، و بازگرداندنِ خطاهای معنادار به شکل پیام‌هایی برای نمایشِ کاربر است.

\subsection{رندر نقشه و کنترل‌ها}
\paragraph{\lr{MapContainer} و \lr{TileLayer}}
\lr{MapContainer} عنصرِ محوریِ رندرِ نقشه است و \lr{TileLayer} منبعِ کاشی‌های نقشه (مثلِ \lr{OpenStreetMap}) را فراهم می‌کند. مرجعِ نقشه در \lr{mapRef} ذخیره می‌شود تا عملیات‌هایی مانند تغییرِ مرکز، بزرگ‌نمایی یا فراخوانیِ توابعِ کتابخانه به‌آسانی انجام شود. مرکزِ نقشه معمولاً بر اساسِ اولین \lr{Measurement} یا انتخابِ کاربر مقداردهی می‌شود تا نمایشِ اولیه معنادار باشد.

\paragraph{کنترل مرکز نقشه}
برای حرکتِ نرم و هماهنگِ نقشه هنگامِ انتخابِ یک رکورد، یک کنترل‌کنندهٔ مبتنی بر هوکِ \lr{useMap} تغییراتِ \lr{center} را شنود می‌کند و از متدِ \lr{flyTo} یا معادلِ آن برای انتقالِ روانِ نما استفاده می‌کند. این رفتار باعث می‌شود که کاربر هنگامِ کلیک بر روی لیستی از رکوردها یا جستجو تجربهٔ دیداریِ پیوسته‌ای داشته باشد.

\paragraph{دیالوگِ تنظیمات و \lr{legend}}
دیالوگِ تنظیمات نقطهٔ ورودِ کاربر به سفارشی‌سازیِ نحوهٔ نمایش است: انتخابِ دستهٔ متریک (\lr{metricType})، انتخابِ متریکِ دقیق، تنظیمِ آستانه‌ها و ویرایشِ طیفِ رنگی . \lr{legend} به‌عنوان راهنمای بصری در صفحه قرار می‌گیرد و بازخوردِ آنی از تنظیماتِ کاربر نشان می‌دهد؛ بنابراین کاربر بلافاصله تأثیرِ تغییرات را بر روی نشانگرها می‌بیند.

\subsection{منطق تعیین رنگ و نمایش نشانگرها}
هدفِ اصلیِ منطقِ رنگ‌دهی این است که مقدارِ عددیِ هر \lr{Measurement} را به یک بازنمایِ بصریِ خوانا و معنی‌دار تبدیل کند. به‌طور مفهومی:
\begin{enumerate}
  \item \textbf{انتخاب مقدارِ معنادار:} برای هر متریک، از میانِ فیلدهای موجود مقدارِ قابل‌اعتماد و مرتبط انتخاب می‌شود (مثلاً برای \lr{signal\_strength} ممکن است چند فیلدِ جایگزین وجود داشته باشد).
  \item \textbf{نرمال‌سازی نسبت به آستانه‌ها:} مقدارِ خام نسبت به آستانه‌های \lr{min}/\lr{mid}/\lr{max} سنجیده و به یک مقیاسِ معمولی (مثلاً \(0\) تا \(1\)) تبدیل می‌شود تا منطقِ رنگ‌دهی برای همهٔ متریک‌ها قابلِ مقایسه شود.
  \item \textbf{نگاشتِ عددی به طیفِ رنگی:} نسبتِ نرمال‌شده روی طیفی بین رنگِ «ضعیف» تا «عالی» نگاشت می‌شود. برای ایجادِ تغییرِ نرم از یک رنگ به رنگِ بعدی از یک مکانیزمِ میان‌یابی (interpolation) استفاده می‌شود تا مرزها مصنوعی و سخت نباشند.
  \item \textbf{حالت‌های معکوس:} برای متریک‌هایی که مقدارِ کمتر بهتر است (مثلاً \lr{latency}) پیش از نرمال‌سازی معکوس‌سازی انجام می‌شود تا قانونِ یکپارچهٔ رنگ‌دهی حفظ شود.
  \item \textbf{مقادیر sentinel و دادهٔ ناموجود:} مقادیرِ مخصوص (مثلِ \verb|-1| یا \verb|null|) به‌عنوان دادهٔ ناموجود شناسایی شده و با یک رنگ/آیکونِ مشخص یا پنهان‌سازیِ هوشمند نمایش داده می‌شوند تا کاربر گمراه نشود.
\end{enumerate}

\subsection{توابع کمکیِ کلیدی}
توابع کمکی جهتِ پشتیبانی از تجربهٔ کاربریِ قابل‌تنظیم و سازگار وجود دارند؛ بدون آوردنِ پیاده‌سازیِ دقیق، شرحِ عملکردِ هر کدام چنین است:
\begin{itemize}
    \item \lr{getMinRange}: مشخص می‌کند کمترین مقدار منطقی برای هر متریک چه‌قدر است—این مقدار پایهٔ حدِ چپِ اسلایدرِ آستانه است.
    \item \lr{getMaxRange}: مشخص می‌کند بیشترین مقدار منطقی برای هر متریک چه‌قدر است—محدودهٔ سمت راستِ اسلایدر.
    \item \lr{getStepSize}: تعیین‌کنندهٔ کوچک‌ترین واحد تغییرِ ممکن در اسلایدر است (مثلاً برای برخی متریک‌ها گامِ 0.1 منطقی‌تر است و برای برخی گامِ 1).
    \item \lr{getContrastColor}: تابعی که به‌صورت مفهومی تعیین می‌کند متنِ روی یک رنگِ پس‌زمینه بهتر است تیره باشد یا روشن تا خوانایی حفظ شود.
\end{itemize}

\subsection{جریانِ داده}
\begin{enumerate}
  \item صفحهٔ \lr{Map} درخواستِ داده را از طریقِ \lr{MapManager} ارسال می‌کند.
  \item پاسخِ سرور توسطِ لایهٔ \lr{MapManager} بررسی و (در صورت نیاز) نرمال‌سازی می‌شود.
  \item داده‌های نرمال‌شده به مجموعهٔ نشانگرها داده می‌شوند تا هرکدام بر اساسِ پیکربندیِ فعلی رنگ و محتوای \lr{Popup} خود را تعیین کنند.
  \item تغییراتِ پیکربندی (آستانه/رنگ/متریک) که توسطِ کاربر انجام می‌شود، بلافاصله به نمایِ نقشه منعکس می‌گردد و در \lr{state} محلیِ صفحه نگهداری می‌شود.
\end{enumerate}

    \section{مدیریت کاربران و سطح دسترسی}
    \subsection{مدیریت منطق تجاری (\lr{Business Logic})}
    \begin{itemize}
    	\item \textbf{managers/UserListManager.js}:
    	این ماژول رابط کاربری برای مدیریت کاربران است.
    	تابع getAll لیست کامل کاربران را از API دریافت می‌کند.
    	توابع ban و allow برای مسدود و فعال‌سازی کاربران با ارسال اطلاعات شماره یا ایمیل به API استفاده می‌شوند.
    	در صورت خطا، پیام مناسب به صورت رشته‌ی JSON یا متن خطا برگردانده می‌شود.
    \end{itemize}
    
    \subsection{هوک‌های سفارشی (\lr{Custom Hooks})}
    \begin{itemize}
    	\item \textbf{hooks/useUserList.js}:
    	این هوک با استفاده از react-query عملیات واکشی، مسدودسازی و رفع مسدودسازی کاربران را مدیریت می‌کند.
    	از useQuery برای دریافت لیست کاربران و از useMutation برای اجرای عملیات مسدودسازی و رفع مسدودسازی استفاده شده است.
    	در هر عملیات، داده‌های کش‌شده به‌روزرسانی و در صورت خطا، تغییرات برگشت داده می‌شود.
    	پیغام‌های موفقیت و خطا با استفاده از react-toastify نمایش داده می‌شوند.
    \end{itemize}
    
    \subsection{لایه رابط کاربری (\lr{UI Layer})}
    \begin{itemize}
    	\item \textbf{pages/UserList/index.jsx}:
    	این کامپوننت لیست کاربران را نمایش می‌دهد و از هوک\\ useUserList برای دریافت داده‌ها استفاده می‌کند.
    	دارای قابلیت جستجو بر اساس ایمیل، شماره تلفن و نام کاربری است و نتیجه را فیلتر می‌کند.
    	نمایش کاربران در جدول همراه با وضعیت تأیید و مسدود بودن آن‌ها با سوئیچ تعاملی برای فعال/مسدود کردن.
    	دارای صفحه‌بندی با تعداد سطر قابل تنظیم است که برای موبایل بهینه شده است.
    	در هنگام بارگذاری داده، نمایش انیمیشن لودینگ و نمایش پیام خطا در صورت عدم موفقیت دریافت داده‌ها.
    	استفاده از قابلیت واکنش‌گرایی برای نمایش مناسب در موبایل با استفاده از \lr{Media Query}.
    	
    	
    \end{itemize}
    