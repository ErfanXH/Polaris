\def \Subject {تمرین چهارم}
\def \Session {4}
\def \Author {مهران رزاقی - حامد سادات - عرفان همتی}
\def \Date {1404/05/31}
\setcounter{chapter}{\Session}

\chapter*{}

\section*{مقدمه}

\section*{1. معماری clean}

\section*{2. در توصیف دیانت پفیوز}

\section*{3. در توصیف لاجوردی پفیوز}

\newpage
\section*{1. DTLS}
\subsection*{DTLS چیست؟}

\textbf{امنیت لایه انتقال دیتاگرام (DTLS)} پروتکلی است که برای فراهم کردن ارتباط امن بر روی پروتکل‌های مبتنی بر دیتاگرام مانند \textbf{UDP} طراحی شده است. این پروتکل نسخه‌ای تطبیق‌یافته از \textbf{TLS (امنیت لایه انتقال)} است که در محیط‌هایی که قابلیت اطمینان تضمین نمی‌شود، کار می‌کند. DTLS سطح امنیتی مشابه TLS را تضمین می‌کند و در عین حال ارتباط کم‌تاخیر را حفظ می‌نماید که برای برنامه‌های بلادرنگ مانند تماس‌های VoIP، پخش ویدیو و بازی‌های آنلاین حیاتی است.

\subsection*{ویژگی‌های کلیدی DTLS}
\begin{enumerate}
    \item \textbf{قابلیت اطمینان}: DTLS بر روی پروتکل‌های انتقال غیرقابل اطمینان مانند UDP عمل می‌کند، اما مکانیزم‌هایی برای مدیریت از دست رفتن، ترتیب‌بندی مجدد و تکرار بسته‌ها فراهم می‌کند.
    \item \textbf{کم‌تاخیر بودن}: برخلاف TLS مبتنی بر TCP، DTLS برای برنامه‌های حساس به زمان مناسب است زیرا سربار اتصال‌های قابل اطمینان را حذف می‌کند.
    \item \textbf{امنیت}: این پروتکل رمزنگاری، احراز اصالت پیام و یکپارچگی را ارائه می‌دهد و محرمانگی و حفاظت در برابر دستکاری را تضمین می‌کند.
\end{enumerate}

\subsection*{جریان پیام‌های DTLS}

فرآیند مذاکره در DTLS مشابه TLS است اما برای محیط دیتاگرام تطبیق یافته است. در ادامه خلاصه‌ای از جریان پیام‌های DTLS آمده است:

\subsection*{1. \lr{ClientHello}}
\begin{itemize}
    \item کلاینت یک پیام \lr{ClientHello} به سرور ارسال می‌کند.
    \item این پیام شامل:
    \begin{itemize}
        \item مجموعه رمزنگاری‌های پشتیبانی شده،
        \item روش‌های فشرده‌سازی،
        \item مقدار تصادفی (برای تولید کلید)،
        \item یک کوکی (اختیاری برای کاهش حملات \lr{DoS}).
    \end{itemize}
\end{itemize}

\subsection*{2. \lr{HelloVerifyRequest} (اختیاری)}
\begin{itemize}
    \item اگر سرور بخواهد توانایی کلاینت برای دریافت بسته‌ها در آدرس IP ادعایی‌اش را تأیید کند، یک پیام \lr{HelloVerifyRequest} ارسال می‌کند.
    \item سرور یک کوکی بدون وضعیت در پاسخ شامل می‌کند.
    \item کلاینت \lr{ClientHello} را همراه با کوکی ارسال می‌کند.
\end{itemize}

\subsection*{3. \lr{ServerHello}}
\begin{itemize}
    \item سرور با یک پیام \lr{ServerHello} پاسخ می‌دهد که شامل:
    \begin{itemize}
        \item مجموعه رمزنگاری انتخاب شده،
        \item شناسه نشست،
        \item یک مقدار تصادفی.
    \end{itemize}
\end{itemize}

\subsection*{4. گواهی‌نامه (\lr{Certificate}) (اختیاری)}
\begin{itemize}
    \item اگر پروتکل رمزنگاری انتخاب شده نیاز داشته باشد، سرور گواهی خود را برای احراز هویت ارسال می‌کند.
\end{itemize}

\subsection*{5. تبادل کلید سرور (\lr{ServerKeyExchange}) (اختیاری)}
\begin{itemize}
    \item اگر مجموعه رمزنگاری نیاز داشته باشد (مانند دیفی-هلمن یا \lr{ECDHE})، سرور پارامترهای تبادل کلید را ارسال می‌کند.
\end{itemize}

\subsection*{6. درخواست گواهی (\lr{CertificateRequest}) (اختیاری)}
\begin{itemize}
    \item سرور از کلاینت درخواست می‌کند گواهی خود را برای احراز هویت متقابل ارسال کند.
\end{itemize}

\subsection*{7. \lr{ServerHelloDone}}
\begin{itemize}
    \item سرور اعلام می‌کند که ارسال پیام‌های خود در مذاکره به پایان رسیده است.
\end{itemize}

\subsection*{8. گواهی‌نامه کلاینت (\lr{Client Certificate}) (اختیاری)}
\begin{itemize}
    \item اگر سرور درخواست گواهی کرده باشد، کلاینت گواهی خود را ارائه می‌دهد.
\end{itemize}

\subsection*{9. تبادل کلید کلاینت (\lr{ClientKeyExchange})}
\begin{itemize}
    \item کلاینت بخش خود از تبادل کلید را به سرور ارسال می‌کند.
\end{itemize}

\subsection*{10. تأیید گواهی (\lr{CertificateVerify}) (اختیاری)}
\begin{itemize}
    \item اگر احراز هویت متقابل استفاده شود، کلاینت مالکیت کلید خصوصی مرتبط با گواهی خود را اثبات می‌کند.
\end{itemize}

\subsection*{11. اتمام (\lr{Finished})}
\begin{itemize}
    \item کلاینت و سرور پیام \lr{Finished} را ارسال می‌کنند تا مذاکره را کامل و امن اعلام کنند.
\end{itemize}

\subsection*{تفاوت‌های جریان پیام DTLS و TLS}
\begin{itemize}
    \item \textbf{\lr{HelloVerifyRequest}}: DTLS این مرحله را برای کاهش حملات \lr{DoS} اضافه کرده است.
    \item \textbf{ارسال مجدد}: DTLS ارسال مجدد پیام‌ها را در صورت از دست رفتن بسته‌ها مدیریت می‌کند.
    \item \textbf{شماره ترتیب صریح}: هر پیام در DTLS شامل یک شماره ترتیب صریح برای ترتیب‌بندی و حفاظت در برابر تکرار است.
    \item \textbf{تکه‌بندی}: DTLS تکه‌بندی صریح پیام‌ها را برای مدیریت اندازه پیام‌هایی که از MTU پروتکل انتقال فراتر می‌روند، پشتیبانی می‌کند.
\end{itemize}
\subsection*{دیاگرام جریان پیام DTLS}
\begin{tikzpicture}[
    font=\footnotesize,
    node distance=1.5cm and 4cm,
    every node/.style={align=center},
    arrow/.style={-{Latex}, thick},
]

% Nodes
\node (client) at (0,0) {Client};
\node (server) at (6,0) {Server};

% Extend vertical lines
\draw[line] (0,-0.5) -- (0,-18.5);
\draw[line] (6,-0.25) -- (6,-18.5);

% Messages
\draw[arrow] (client) -- ++(0,-1.5) node[midway,right] {ClientHello} -- ++(6,0);
\draw[arrow] (server) ++(0,-1.5) -- ++(0,-1.5) node[midway,left] {HelloVerifyRequest} -- ++(-6,0);
\draw[arrow] (client) ++(0,-3) -- ++(0,-1.5) node[midway,right] {ClientHello} -- ++(6,0);
\draw[arrow] (server) ++(0,-4.5) -- ++(0,-1.5) node[midway,left] {ServerHello} -- ++(-6,0);
\draw[arrow] (server) ++(0,-6) -- ++(0,-1.5) node[midway,left] {Certificate} -- ++(-6,0);
\draw[arrow] (server) ++(0,-7.5) -- ++(0,-1.5) node[midway,left] {ServerKeyExchange} -- ++(-6,0);
\draw[arrow] (server) ++(0,-9) -- ++(0,-1.5) node[midway,left] {ServerHelloDone} -- ++(-6,0);
\draw[arrow] (client) ++(0,-10.5) -- ++(0,-1.5) node[midway,right] {ClientKeyExchange} -- ++(6,0);
\draw[arrow] (client) ++(0,-12) -- ++(0,-1.5) node[midway,right] {ChangeCipherSpec} -- ++(6,0);
\draw[arrow] (client) ++(0,-13.5) -- ++(0,-1.5) node[midway,right] {Finished} -- ++(6,0);
\draw[arrow] (server) ++(0,-15) -- ++(0,-1.5) node[midway,left] {ChangeCipherSpec} -- ++(-6,0);
\draw[arrow] (server) ++(0,-16.5) -- ++(0,-1.5) node[midway,left] {Finished} -- ++(-6,0);

\end{tikzpicture}

\subsection*{موارد استفاده DTLS}
\begin{itemize}
    \item \textbf{VoIP}: اطمینان از ارتباط امن در تماس‌های صوتی بلادرنگ.
    \item \textbf{پخش ویدیو}: تحویل ویدیوی کم‌تاخیر با رمزنگاری.
    \item \textbf{اینترنت اشیا (IoT)}: تأمین امنیت ارتباط در دستگاه‌های سبک‌وزن.
    \item \textbf{VPN}: تونل‌سازی امن بر روی UDP در پیاده‌سازی‌های VPN مانند \lr{OpenVPN}.
\end{itemize}

DTLS تعادلی بین امنیت TLS و مزایای عملکردی UDP ارائه می‌دهد و آن را برای ارتباطات بلادرنگ، سبک و شبکه‌های غیرقابل اطمینان ایده‌آل می‌سازد.

\section*{2. حمله Downgrade در TLS}
\subsection*{حمله کاهش رتبه در TLS}

حمله کاهش رتبه \lr{(Downgrade Attack)} زمانی رخ می‌دهد که یک مهاجم، کلاینت و سرور را مجبور می‌کند از نسخه‌ای قدیمی‌تر و کم‌امن‌تر از پروتکل یا الگوریتم‌های رمزنگاری ضعیف‌تر در فرآیند \lr{Handshake} استفاده کنند. این موضوع به مهاجم اجازه می‌دهد از ضعف‌های موجود در پروتکل‌ها یا رمزنگاری‌های قدیمی سوءاستفاده کرده و محرمانگی و یکپارچگی ارتباط را به خطر بیاندازد.

\subsection*{انواع حملات کاهش رتبه}

\begin{enumerate}
    \item \textbf{کاهش رتبه پروتکل}
        \begin{itemize}
            \item مهاجم \lr{Handshake} را دستکاری می‌کند تا کلاینت و سرور به نسخه‌ای قدیمی‌تر و ناامن‌تر از \lr{TLS} بازگردند (مثلاً از \lr{TLS 1.3} به \lr{TLS 1.2} یا حتی \lr{SSL 3.0}).
            \item مثال: حمله \lr{POODLE} (\lr{Padding Oracle On Downgraded Legacy Encryption}) از بازگشت به \lr{SSL 3.0} سوءاستفاده کرد.
        \end{itemize}

    \item \textbf{کاهش رتبه مجموعه رمزنگاری \lr{(Cipher Suite)}}
        \begin{itemize}
            \item مهاجم استفاده از الگوریتم‌های رمزنگاری ضعیف‌تر (مثل \lr{DES} یا \lr{RC4}) را تحمیل می‌کند، حتی اگر الگوریتم‌های قوی‌تر موجود باشند.
            \item مثال: حمله \lr{Logjam} از کلیدهای \lr{Diffie-Hellman} ضعیف با درجه صادرات سوءاستفاده کرد.
        \end{itemize}

    \item \textbf{کاهش رتبه تبادل کلید}
        \begin{itemize}
            \item مهاجم کلاینت و سرور را فریب می‌دهد تا از مکانیزم تبادل کلید ضعیف‌تر (مثل \lr{RSA} استاتیک یا \lr{Diffie-Hellman} با درجه صادرات) استفاده کنند.
        \end{itemize}

    \item \textbf{کاهش رتبه \lr{Handshake}}
        \begin{itemize}
            \item مهاجم پیام‌های \lr{Handshake} را مختل یا تغییر می‌دهد تا تنظیمات امنیتی بین کلاینت و سرور را کاهش دهد.
        \end{itemize}
\end{enumerate}

\subsection*{چگونه حملات کاهش رتبه کار می‌کنند؟}

\begin{enumerate}
    \item \textbf{جاسوسی در \lr{Handshake}}
        \begin{itemize}
            \item مهاجم به عنوان یک میانجی (\lr{MITM}) عمل کرده و \lr{Handshake} بین کلاینت و سرور را شنود می‌کند.
        \end{itemize}

    \item \textbf{تحمیل کاهش رتبه}
        \begin{itemize}
            \item با دستکاری پیام‌های \lr{Handshake}، مهاجم یک یا هر دو طرف را متقاعد می‌کند تا از نسخه یا مجموعه رمزنگاری کمتر امن استفاده کنند.
        \end{itemize}

    \item \textbf{سوءاستفاده از ضعف‌ها}
        \begin{itemize}
            \item پس از برقراری جلسه با امنیت پایین‌تر، مهاجم از ضعف‌های موجود در پروتکل یا رمزنگاری کاهش یافته بهره‌برداری می‌کند.
        \end{itemize}
\end{enumerate}

\subsection*{روش‌های پیشگیری از حملات کاهش رتبه}

\begin{enumerate}
    \item \textbf{اجرای نسخه‌های امن \lr{TLS}}
        \begin{itemize}
            \item کلاینت‌ها و سرورها باید نسخه‌های قدیمی و ناامن \lr{TLS} (مانند \lr{SSL 3.0}، \lr{TLS 1.0} و \lr{TLS 1.1}) را غیرفعال کنند.
            \item فقط از نسخه‌های امن مانند \lr{TLS 1.2} و \lr{TLS 1.3} استفاده شود.
        \end{itemize}

    \item \textbf{پیشگیری از کاهش رتبه در سرور}
        \begin{itemize}
            \item از مکانیسم‌هایی مانند \lr{TLS\_FALLBACK\_SCSV} استفاده شود.
            \item در صورت تشخیص بازگشت، سرور اتصال را رد کند.
        \end{itemize}

    \item \textbf{استفاده از مجموعه رمزنگاری قوی}
        \begin{itemize}
            \item مجموعه‌های رمزنگاری ضعیف (مانند \lr{RC4} و رمزهای صادراتی) غیرفعال شوند.
            \item تنها مجموعه‌های مدرن و قوی مانند \lr{AES-GCM} یا \lr{ChaCha20} مجاز باشند.
        \end{itemize}

    \item \textbf{امنیت سخت‌افزاری حمل و نقل}
        \begin{itemize}
            \item از \lr{HSTS} برای اطمینان از ارتباط فقط بر بستر \lr{HTTPS} استفاده شود.
        \end{itemize}

    \item \textbf{هش کردن جلسات (برای \lr{TLS 1.2} و نسخه‌های قبلی)}
        \begin{itemize}
            \item از افزونه‌های هش جلسات برای جلوگیری از دستکاری پیام‌های \lr{Handshake} استفاده شود.
        \end{itemize}

    \item \textbf{استفاده از \lr{TLS 1.3}}
        \begin{itemize}
            \item \lr{TLS 1.3} بسیاری از ویژگی‌های قدیمی را حذف کرده و از مکانیسم‌های بازگشت ناامن پشتیبانی نمی‌کند.
        \end{itemize}

    \item \textbf{پیکربندی کلاینت و سرور}
        \begin{itemize}
            \item کلاینت و سرور باید به گونه‌ای تنظیم شوند که بالاترین نسخه پشتیبانی‌شده از \lr{TLS} را بدون بازگشت به نسخه‌های ناامن مذاکره کنند.
        \end{itemize}

    \item \textbf{بروزرسانی منظم و مدیریت وصله‌ها}
        \begin{itemize}
            \item کتابخانه‌ها و برنامه‌های \lr{TLS} را بروزرسانی کرده و آسیب‌پذیری‌ها را رفع کنید.
        \end{itemize}
\end{enumerate}

\subsection*{نمونه واقعی: حمله \lr{POODLE}}
\begin{itemize}
    \item حمله \lr{POODLE} (2014) از بازگشت به \lr{SSL 3.0} سوءاستفاده کرد.
    \item \lr{SSL 3.0} دارای آسیب‌پذیری‌هایی در پدینگ بود که به مهاجمان اجازه می‌داد داده‌ها را رمزگشایی کنند.
    \item پیشگیری شامل غیرفعال کردن \lr{SSL 3.0} و اجرای \lr{TLS\_FALLBACK\_SCSV} بود.
\end{itemize}


\section*{3. گواهی ریشه چیست؟}
گواهی ریشه یک گواهی دیجیتال است که توسط یک \lr{Certificate Authority} یا به اختصار \lr{CA} صادر و امضا شده است. این گواهی در رأس زنجیره اعتماد (\lr{Chain of Trust}) قرار دارد و تمام گواهی‌های دیگر در این زنجیره مانند گواهی‌های میانی و گواهی‌های نهایی به آن اعتماد می‌کنند. گواهی‌های ریشه معمولاً به‌صورت پیش‌فرض در سیستم‌عامل‌ها (مانند ویندوز، لینوکس یا مک‌اواس) و مرورگرها (مانند کروم و فایرفاکس) نصب می‌شوند.

تأیید گواهی ریشه از طریق مراحل زیر انجام می‌شود:
\begin{itemize}
    \item \textbf{وجود در لیست گواهی‌های معتبر:} سیستم‌عامل یا مرورگر، گواهی ریشه را با لیست داخلی گواهی‌های معتبر خود مقایسه می‌کند. اگر گواهی در این لیست وجود داشته باشد، به آن اعتماد می‌شود.
    \item \textbf{زنجیره اعتماد:} مرورگر یا سیستم‌عامل بررسی می‌کند که آیا گواهی مورد استفاده توسط یک گواهی میانی معتبر امضا شده و گواهی میانی به گواهی ریشه متصل است یا خیر.
    \item \textbf{اعتبار زمانی:} تاریخ انقضای گواهی بررسی می‌شود. اگر گواهی منقضی شده باشد، دیگر معتبر نیست.
    \item \textbf{عدم ابطال:} سیستم از طریق لیست‌های ابطال گواهی (\lr{CRL}) یا پروتکل \lr{OCSP} بررسی می‌کند که آیا گواهی ابطال شده است یا خیر.
\end{itemize}

\subsection*{4. \lr{Samy Kamkar}}

سامی کامکار یک هکر، برنامه‌نویس و محقق امنیت سایبری مشهور است که با خلاقیت‌ها و فعالیت‌های جنجالی خود در حوزه فناوری شناخته می‌شود. او در سال ۲۰۰۵ با ایجاد کرم مای‌اسپیس که به سرعت در کمتر از ۲۴ ساعت میلیون‌ها حساب کاربری را آلوده کرد، به شهرت جهانی دست یافت. این کرم توانست رکوردهای بسیاری را بشکند و نشان‌دهنده قدرت آسیب‌پذیری‌های موجود در شبکه‌های اجتماعی باشد. این رویداد باعث شد سامی با چالش‌های قانونی جدی روبرو شود، اما در نهایت او مسیر حرفه‌ای خود را به سمت فعالیت‌های مثبت و سازنده در زمینه امنیت اطلاعات تغییر داد.

کامکار ابزارها و پروژه‌های متعددی را در حوزه امنیت سایبری توسعه داده است. یکی از معروف‌ترین اختراعات او، دستگاهی به نام "Evercookie" است که یک نوع کوکی بسیار مقاوم است و نشان‌دهنده چگونگی سوءاستفاده از داده‌های کاربری در وب می‌باشد. او همچنین دستگاه "KeySweeper" را طراحی کرد که می‌تواند کلیدهای تایپ‌شده بر روی کیبوردهای بی‌سیم را رهگیری کند.

سامی به عنوان یک پیشگام در زمینه هک اخلاقی شناخته می‌شود و تمرکز خود را بر افزایش آگاهی عمومی درباره خطرات امنیتی و ارائه راهکارهای ایمن‌سازی فناوری‌ها قرار داده است. او در کنفرانس‌های معتبر امنیت سایبری سخنرانی کرده و در رسانه‌های مختلف به عنوان یک متخصص امنیت حضور یافته است. سامی کامکار نماد فردی است که از مهارت‌های فنی خود برای آشکار کردن نقاط ضعف سیستم‌ها و ارتقاء امنیت جهانی استفاده می‌کند.

\newpage
\section*{1. \lr{TLS 1.3}}
\subsection*{استخراج پیام ها و رمزگشایی آنها}
در ابتدا لازم است با استفاده از دستور \lr{export SSLKEYLOGFILE='some\_address'} کلیدهای SSL را به یک فایل دلخواه منتقل کنیم.\\
\includegraphics[width=\linewidth]{images/export.png}
حال می بایست با استفاده از دستور \lr{sudo wireshark} نرم افزار wireshark را باز کنیم.\\
\includegraphics[width=\linewidth]{images/wireshark.png}
در wireshark باز شده باید تنظیمات TLS موجود در آدرس \lr{Edit>Preferences>Protocols>TLS} بیابیم. سپس آدرس فایل دلخواه را در فیلد \lr{(Pre)-Master-Secret log filename} قرار می دهیم.
\includegraphics[width=\linewidth]{images/preferences.png}
حال می توانیم یک ترمینال جدید باز کرده و یک اتصال به یک سرور با استفاده از \lr{TLS 1.3} برای مثال در اینجا nodejs ایجاد کنیم. پیش از برقراری این اتصال لازم است که capture در wireshark فعال شود تا بسته های دریافت شده نشان داده شوند.\\
\includegraphics[width=\linewidth]{images/curl.png}
حال نتیجه این اتصال 125 فریم بوده است که قسمتی از آن در ادامه آورده شده است:\\
\includegraphics[width=\linewidth]{images/Frame1-28.png}
در ادامه چهار پیام TLS تبادل شده شامل \lr{Client Hello}، \lr{Server Hello, Change Cipher Spec}،\\ \lr{Encrypted Extensions, Certificate, Certificate Verify, Finished}، \lr{Change Cipher Spec, Finished} توضیح داده می شود.
\subsection*{\lr{Client Hello}}
این بسته شامل پیام \lr{"Client Hello"} از یک handshake پروتکل TLS می‌باشد. این پیام در ایجاد یک ارتباط امن بین مشتری و سرور اهمیت زیادی دارد. در زیر بخش‌های مربوطه مورد بررسی قرار می‌گیرد:

\subsection*{اطلاعات کلی بسته}
\begin{itemize}[label={--}]
    \item \textbf{اطلاعات فریم}:
    \begin{itemize}
        \item \textbf{شماره فریم}: ۸
        \item \textbf{اندازه کل}: ۵۷۱ بایت
        \item \textbf{زمان ورود}: این بسته در تاریخ ۲۹ دسامبر ۲۰۲۴ ساعت ۱۲:۲۴:۵۶ وارد شد.
        \item \textbf{پروتکل‌ها}: پروتکل‌های محصور شده شامل Ethernet، IPv4، TCP و TLS هستند.
    \end{itemize}
\end{itemize}

\subsection*{لایه Ethernet}
\begin{itemize}[label={--}]
    \item \textbf{آدرس‌های MAC منبع و مقصد}:
    \begin{itemize}
        \item \textbf{منبع}: 08:00:27:31:3d:eb (PcsCompu)
        \item \textbf{مقصد}: 52:54:00:12:35:00 (RealtekU)
    \end{itemize}
    \item \textbf{نوع}: این فریم از IPv4 (0x0800) استفاده می‌کند.
\end{itemize}

\subsection*{لایه IP}
\begin{itemize}[label={--}]
    \item \textbf{آدرس IP منبع}: 10.0.2.6
    \item \textbf{آدرس IP مقصد}: 104.20.23.46
    \item \textbf{پروتکل}: پروتکل انتقال TCP (6).
    \item \textbf{طول کل}: کل بسته IP ۵۵۷ بایت است.
\end{itemize}

\subsection*{لایه TCP}
\begin{itemize}[label={--}]
    \item \textbf{پورت منبع}: 38420
    \item \textbf{پورت مقصد}: 443 (پورت استاندارد برای \lr{HTTPS}).
    \item \textbf{شماره توالی}: ۱
    \item \textbf{شماره تأیید}: ۱
    \item \textbf{پرچم‌ها}: پرچم‌های TCP نشان می‌دهند که این یک بسته Push (PSH) و تأیید (ACK) است.
    \item \textbf{طول Payload TCP}: payload بخش TCP ۵۱۷ بایت است که شامل داده‌های TLS می‌باشد.
\end{itemize}

\subsection*{لایه TLS}
\begin{itemize}[label={--}]
    \item \textbf{نوع محتوا}: این یک پیام Handshake (نوع ۲۲) است.
    \item \textbf{نسخه TLS}: سازگاری با TLS 1.0 (0x0301) را نشان می‌دهد، اما برای handshake از TLS 1.2 (0x0303) استفاده می‌شود.
    \item \textbf{تصادفی}: یک مقدار تصادفی ۳۲ بایتی برای تولید کلید نشست استفاده می‌شود.
    \item \textbf{شناسه نشست}: یک شناسنامه منحصر به فرد برای نشست (۳۲ بایت).
    \item \textbf{مجموعه‌های رمزنگاری}:
    \begin{itemize}
        \item پشتیبانی از چهار مجموعه رمزنگاری شامل:
        \begin{itemize}
            \item TLS\_AES\_256\_GCM\_SHA384
            \item TLS\_CHACHA20\_POLY1305\_SHA256
            \item TLS\_AES\_128\_GCM\_SHA256
            \item TLS\_EMPTY\_RENEGOTIATION\_INFO\_SCSV (برای سازگاری).
        \end{itemize}
    \end{itemize}
    \item \textbf{روش‌های فشرده‌سازی}: فقط روش null پشتیبانی می‌شود (به معنی عدم فشرده‌سازی).
    \item \textbf{افزونه‌ها}: افزونه‌های مختلف TLS شامل:
    \begin{itemize}
        \item \texttt{server\_name}: nodejs.org (برای استفاده از \lr{Server Name Indication}، \lr{SNI}).
        \item \texttt{supported\_groups}: لیست منحنی‌های بیضوی پشتیبانی‌شده برای تبادل کلید.
        \item \texttt{ALPN} (مذاکره پروتکل لایه کاربردی): نشان‌دهنده پشتیبانی از HTTP/2 (h2) و HTTP/1.1.
        \item \texttt{Key Share}: شامل یک کلید مشترک برای x25519 که برای تبادل کلید استفاده می‌شود.
        \item \texttt{Padding}: برای اطمینان از اینکه طول payload مضربی از اندازه بلوک باشد، استفاده می‌شود.
    \end{itemize}
\end{itemize}

\subsection*{اثر انگشت JA3}
\begin{itemize}[label={--}]
    \item \textbf{JA3}: این یک هش از پارامترهای Client Hello است که می‌تواند برای شناسایی پیاده‌سازی TLS مشتری در تحلیل ترافیک استفاده شود.
\end{itemize}
این بسته Client Hello ارتباط امنی را به سرور در آدرس 104.20.23.46 بر روی پورت 443 آغاز می‌کند و مجموعه‌های رمزنگاری و افزونه‌های مختلفی را برای مذاکره ارائه می‌دهد. استفاده از SNI به سرور اجازه می‌دهد تا گواهی SSL مناسب برای نام میزبان درخواست شده را ارائه دهد و افزونه‌ها نشان می‌دهد که مشتری از شیوه‌های رمزنگاری مدرن پشتیبانی می‌کند.
\subsection*{\lr{Server Hello, Change Cipher Spec}}
این frame شامل اجزای کلیدی فرآیندهای \lr{TLS Handshake} و \lr{Change Cipher Spec} است که پاسخ سرور را در طول handshake توصیف می‌کند. توضیح هر بخش از این پکت در زیر آمده است:

\section*{اطلاعات frame}
\begin{itemize}
    \item \textbf{frame}: 1314 بایت.
    \item \textbf{پروتکل‌ها}: \lr{Ethernet}، \lr{IPv4}، \lr{TCP} و \lr{TLS}.
\end{itemize}

\section*{لایه Ethernet}
\begin{itemize}
    \item \textbf{آدرس مک منبع}: \lr{RealtekU\_12:35:00 (52:54:00:12:35:00)} - کارت شبکه سرور.
    \item \textbf{آدرس مک مقصد}: \lr{PcsCompu\_31:3d:eb (08:00:27:31:3d:eb)} - کارت شبکه کلاینت.
    \item \textbf{نوع}: \lr{IPv4 (0x0800)}.
\end{itemize}

\section*{لایه IP}
\begin{itemize}
    \item \textbf{IP منبع}: \lr{104.20.23.46} - آی‌پی عمومی سرور.
    \item \textbf{IP مقصد}: \lr{10.0.2.6} - آی‌پی خصوصی کلاینت.
    \item \textbf{پروتکل}: \lr{TCP}.
    \item \textbf{اندازه بسته}: 1300 بایت (شامل هدر و محموله).
\end{itemize}

\section*{لایه TCP}
\begin{itemize}
    \item \textbf{پورت منبع}: 443 (\lr{HTTPS}).
    \item \textbf{پورت مقصد}: 38420 (پورت موقت کلاینت).
    \item \textbf{ترتیب و تاییدیه}:
    \begin{itemize}
        \item \textbf{شماره ترتیب}: 1 (نسبت به این اتصال).
        \item \textbf{شماره تأییدیه}: 518 (تأیید داده قبلی کلاینت).
    \end{itemize}
    \item \textbf{پرچم‌ها}: تنظیم \lr{PSH} (فشار) و \lr{ACK} (تأییدیه).
    \item \textbf{طول محموله}: 1260 بایت.
\end{itemize}

\section*{لایه TLS}
\subsection*{\lr{Server Hello}}
\begin{itemize}
    \item \textbf{نوع محتوا}: \lr{Handshake (22)}.
    \item \textbf{نسخه تی‌ال‌اس}: \lr{TLS 1.2 (0x0303)} - برای سازگاری اما مذاکره \lr{TLS 1.3}.
    \item \textbf{مقدار تصادفی}: عدد 32 بایتی تولیدشده توسط سرور، برای تولید کلید.
    \item \textbf{شناسه نشست}: شناسه 32 بایتی منحصربه‌فرد.
    \item \textbf{\lr{Cipher Suite}}: \lr{TLS\_AES\_256\_GCM\_SHA384 (0x1302)} - الگوریتم رمزنگاری و هش.
    \item \textbf{روش فشرده‌سازی}: هیچ (بدون فشرده‌سازی).
    \item \textbf{افزونه‌ها}:
    \begin{itemize}
        \item \textbf{مشارکت کلید}:
        \begin{itemize}
            \item گروه: \lr{x25519} (منحنی بیضوی برای تبادل کلید دیفی-هلمن).
            \item تبادل کلید: کلید عمومی 32 بایتی سرور.
        \end{itemize}
        \item \textbf{نسخه‌های پشتیبانی‌شده}:
        \begin{itemize}
            \item \lr{TLS 1.3 (0x0304)}.
        \end{itemize}
    \end{itemize}
\end{itemize}

\subsection*{تغییر مشخصات رمز}
\begin{itemize}
    \item \textbf{نوع محتوا}: \lr{Change Cipher Spec (20)}.
    \item \textbf{پیام}: اعلام به کلاینت برای استفاده از کلیدهای رمزنگاری.
\end{itemize}

\subsection*{خلاصه}
\begin{enumerate}
    \item \textbf{\lr{Server Hello}}: 
    \begin{itemize}
        \item نسخه \lr{TLS 1.3} و \lr{Cipher Suite} را تأیید می‌کند.
        \item کلید عمومی را برای تبادل امن کلید ارائه می‌دهد.
        \item مقدار تصادفی و شناسه نشست برای مدیریت نشست فراهم می‌کند.
    \end{itemize}
    \item \textbf{تغییر مشخصات رمز}:
    \begin{itemize}
        \item سیگنالی برای شروع رمزنگاری ارتباط.
    \end{itemize}
    \item \textbf{اثر انگشت \lr{JA3S}}:
    \begin{itemize}
        \item هش منحصربه‌فرد \lr{907bf3ecef1c987c889946b737b43de8}.
    \end{itemize}
\end{enumerate}

این frame گذار سرور به ارتباط رمزنگاری‌شده با کلاینت را نشان می‌دهد که گامی حیاتی در فرآیند handshake TLS است.
\subsection*{\lr{Encrypted Extensions, Certificate, Certificate Verify, Finished}}
\subsection*{توضیحات لایه‌ها}

\subsection*{1. لایه Ethernet}
\begin{itemize}
    \item \textbf{آدرس MAC مبدأ:} \lr{RealtekU\_12:35:00 (52:54:00:12:35:00)} — آدرس فیزیکی سرور یا دروازه واسط را نشان می‌دهد.
    \item \textbf{آدرس MAC مقصد:} \lr{PcsCompu\_31:3d:eb (08:00:27:31:3d:eb)} — آدرس فیزیکی ماشین محلی شما را نشان می‌دهد.
    \item این فریم تضمین می‌کند که داده‌ها به ماشین درست در شبکه محلی منتقل شوند.
\end{itemize}

\subsection*{2. لایه IP (IPv4)}
\begin{itemize}
    \item \textbf{IP مبدأ:} \lr{104.20.23.46} — آدرس آی‌پی سرور راه دور.
    \item \textbf{IP مقصد:} \lr{10.0.2.6} — آدرس آی‌پی محلی/داخلی شما.
    \item \textbf{پروتکل:} TCP (\lr{6}).
    \item لایه IP مسئول تحویل انتها به انتها داده‌ها است.
\end{itemize}

\subsection*{3. لایه TCP}
\begin{itemize}
    \item \textbf{پورت مبدأ:} \lr{443} — پورت استاندارد برای \lr{HTTPS}.
    \item \textbf{پورت مقصد:} \lr{38420} — پورتی موقت که توسط کلاینت شما استفاده می‌شود.
    \item \textbf{شماره توالی:} \lr{4181} — موقعیت اولین بایت در این بخش را نشان می‌دهد.
    \item \textbf{شماره تأیید:} \lr{518} — دریافت داده‌های قبلی از کلاینت را تأیید می‌کند.
    \item \textbf{پرچم‌ها:} \lr{PSH} (ارسال داده) و \lr{ACK} (تأیید).
    \item \textbf{طول بارگذاری:} \lr{1401 bytes} — پیام‌های ارتباط \lr{TLS} در این بارگذاری \lr{TCP} قرار دارند.
\end{itemize}
\subsection*{4. لایه TLS}
\subsection*{1. الحاقات رمزگذاری شده}
\textbf{هدف:} پیام الحاقات رمزگذاری شده در handshake پروتکل TLS برای انتقال اطلاعات اضافی از کلاینت به سرور پس از پیام ServerHello استفاده می‌شود. این پیام می‌تواند شامل الحاقات مختلفی باشد که قابلیت‌ها یا ترجیحات اضافی برای جلسه TLS را فراهم می‌کند.\\
\textbf{نوع \lr{handshake}:} با مقدار 8 شناسایی شده است، این نشان می‌دهد که پیام بخشی از پروتکل handshake مربوط به الحاقات رمزگذاری شده است.\\
\textbf{طول:} طول این پیام 15 بایت است که شامل فیلدهای الحاقاتی می‌شود.\\
\textbf{الحاقات شامل:}
\begin{itemize}
    \item \texttt{server\_name}: این الحاق می‌تواند نشان دهد که کدام سرور را کلاینت سعی دارد به آن متصل شود. در اینجا، طول آن 0 است، به این معنی که نام سرور خاصی ارائه نشده است.
    \item \texttt{application\_layer\_protocol\_negotiation}: این الحاق به مذاکره در مورد پروتکل‌های لایه کاربردی که باید استفاده شوند کمک می‌کند (مانند \lr{HTTP/2}، \lr{HTTP/1.1}). طول آن 5 است که شامل یک نشانگر نوع و پروتکل‌های واقعی پشتیبانی شده می‌باشد.
\end{itemize}

\subsection*{2. گواهینامه}
\textbf{هدف:} پیام گواهینامه شامل یک یا چند گواهینامه دیجیتال است که از سرور به کلاینت ارسال می‌شود. این برای تعیین هویت سرور بسیار حیاتی است.\\
\textbf{نوع \lr{handshake}:} با مقدار 11 شناسایی شده است، این نشان می‌دهد که پیام بخشی از پروتکل handshake مربوط به گواهینامه‌ها است.\\
\textbf{طول:} طول این پیام 4831 بایت است.\\
\textbf{طول گواهینامه‌ها:} این طول کل گواهینامه‌های موجود در پیام را نشان می‌دهد (4827 بایت).\\
\textbf{جزئیات گواهینامه:}
\begin{itemize}
    \item \textbf{گواهینامه اول:}
        \begin{itemize}
            \item \textbf{نام مشترک:} این نام دامنه‌ای را که گواهینامه برای آن صادر شده است شناسایی می‌کند، در این مورد، \texttt{*.nodejs.org}.
            \item \textbf{صادر کننده:} نهاد صادرکننده گواهینامه را نشان می‌دهد (\lr{Sectigo RSA Domain Validation Secure Server CA}).
            \item \textbf{اعتبار:} دامنه تاریخ معتبر گواهینامه را مشخص می‌کند (از: 2024-02-28، تا: 2025-03-30).
            \item \textbf{الگوریتم کلید عمومی:} برای رمزگذاری از RSA استفاده می‌شود.
            \item \textbf{الحاقات:} الحاقات مختلفی که اطلاعات اضافی در مورد گواهینامه فراهم می‌کند.
        \end{itemize}
    \item \textbf{گواهینامه‌های اضافی:} ساختار مشابهی برای سایر گواهینامه‌ها، که زنجیره اعتماد را تأیید می‌کند.
\end{itemize}

\subsection*{3. تأیید گواهینامه}
\textbf{هدف:} پیام تأیید گواهینامه برای ارائه مدرکی استفاده می‌شود که نشان دهد دارنده کلید خصوصی مربوط به کلید عمومی در گواهینامه کلاینت، همان نهاد است که اتصال را آغاز می‌کند.\\
\textbf{نوع \lr{handshake}:} با مقدار 15 شناسایی شده است، که نشان می‌دهد این پیام یک پیام تأیید گواهینامه است.\\
\textbf{طول:} طول این پیام 516 بایت است.\\
\textbf{الگوریتم امضا:} الگوریتم استفاده شده برای امضای پیام را نشان می‌دهد (\lr{RSA-PSS} با \lr{SHA256}).\\
\textbf{طول امضا:} طول امضا 512 بایت است و امضای واقعی به دنبال آن می‌آید که صحت و اصالت پیام‌های قبلی را تأیید می‌کند.

\subsection*{4. پایان}
\textbf{هدف:} پیام پایان نشان می‌دهد که فرستنده تبادل پیام‌های handshake را تکمیل کرده است. همچنین شامل یک هش از تمامی پیام‌های handshake قبلی برای تأیید صحت handshake است.\\
\textbf{نوع \lr{handshake}:} با مقدار 20 شناسایی شده است، این نشان می‌دهد که این پیام فرایند handshake را به پایان می‌رساند.\\
\textbf{طول:} طول این پیام 48 بایت است.\\
\textbf{داده‌های تأیید:} این داده‌ها بر اساس تمامی پیام‌های قبلی مبادله شده در طول handshake تولید می‌شود و اطمینان می‌دهد که ارتباط دستکاری نشده است.

\subsection*{\lr{Change Cipher Spec, Finished}}
\begin{itemize}
    \item \textbf{Ethernet}:
    \begin{itemize}
        \item \textbf{آدرس \lr{MAC} منبع}: \lr{PcsCompu\_31:3d:eb} (ماشین محلی شما)
        \item \textbf{آدرس \lr{MAC} مقصد}: \lr{RealtekU\_12:35:00} (سرور مقصد یا گیت‌وی میانی)
        \item این آدرس‌ها دستگاه‌های مبدأ و مقصد را در لایه داده مشخص می‌کنند.
    \end{itemize}
    \item \textbf{\lr{IP (IPv4)}}:
    \begin{itemize}
        \item \textbf{آدرس آی‌پی منبع}: \lr{10.0.2.6} (احتمالاً آدرسی داخلی یا محلی)
        \item \textbf{آدرس آی‌پی مقصد}: \lr{104.20.23.46} (سرور راه دور)
        \item \textbf{پروتکل}: \lr{TCP} (مقدار \lr{6})
        \item \textbf{فلگ عدم تکه‌تکه شدن}: تنظیم‌شده، به این معنا که بسته نباید تکه‌تکه شود.
    \end{itemize}
\end{itemize}

\subsection*{۲. لایه \lr{TCP}}
\begin{itemize}
    \item \textbf{پورت منبع}: \lr{38420} (پورت موقتی مورد استفاده کلاینت شما)
    \item \textbf{پورت مقصد}: \lr{443} (پورت استاندارد برای \lr{HTTPS})
    \item \textbf{شماره توالی}: \lr{518} (نسبی)
    \item \textbf{شماره تأییدیه}: \lr{5582} (نسبی)
    \item \textbf{پرچم‌ها}: \lr{PSH} (داده‌ها را فوراً ارسال کن) و \lr{ACK} (تأیید)
    \item \textbf{اندازه پنجره}: \lr{63000} (فضای بافر برای کنترل جریان)
    \item \textbf{طول بار مفید}: ۸۰ بایت (داده \lr{TLS} در بسته \lr{TCP}).
\end{itemize}


\subsection*{۳. لایه \lr{TLS}}
\subsection*{الف. \lr{Change Cipher Spec (CCS)}}
\begin{itemize}
    \item \textbf{نوع محتوا}: \lr{Change Cipher Spec (20)}.
    \item \textbf{نسخه}: \lr{TLS 1.2} (\lr{0x0303}) برای سازگاری با نسخه‌های قبلی، اگرچه این یک handshake \lr{TLS 1.3} است.
    \item \textbf{طول}: ۱ بایت.
    \item \textbf{پیام}: این پیام نشان می‌دهد که پیام‌های بعدی با استفاده از مجموعه رمزنگاری تازه توافق‌شده رمزگذاری خواهند شد. با این حال، در \lr{TLS 1.3}، \lr{CCS} عمدتاً برای سازگاری به عقب گنجانده شده و عملکرد خاصی ندارد.
\end{itemize}

\subsection*{ب. پیام \lr{Finished}}
\begin{itemize}
    \item \textbf{نوع غیرشفاف}: \lr{Application Data (23)}.
    \begin{itemize}
        \item نشان می‌دهد که پیام \lr{Finished} با استفاده از کلیدهای توافق‌شده رمزگذاری شده است.
    \end{itemize}
    \item \textbf{نسخه}: \lr{TLS 1.2} (\lr{0x0303}).
    \item \textbf{طول}: ۶۹ بایت (کل پیام رمزگذاری‌شده).
    \item \textbf{نوع دست‌دهی}: \lr{Finished (20)} (از بار مفید رمزگشایی‌شده).
    \item \textbf{طول محتوای \lr{Finished}}: ۴۸ بایت.
    \item \textbf{داده تأییدیه}:
    \begin{itemize}
        \item این بخش شامل هش رمزنگاری شده‌ای از تمام پیام‌های قبلی handshake است.
        \item تضمین می‌کند که handshake معتبر بوده و هیچ‌گونه تغییر یا دستکاری رخ نداده است.
    \end{itemize}
\end{itemize}


\subsection*{اهمیت این پیام‌ها}
\begin{enumerate}
    \item \textbf{\lr{Change Cipher Spec}}:
    \begin{itemize}
        \item نشان‌دهنده انتقال به استفاده از کلیدهای رمزنگاری توافق‌شده برای رمزگذاری و تأیید پیام‌ها است.
    \end{itemize}
    \item \textbf{\lr{Finished Message}}:
    \begin{itemize}
        \item یکپارچگی handshake را تأیید می‌کند.
        \item اطمینان حاصل می‌کند که هر دو طرف کلیدهای جلسه یکسانی به دست آورده‌اند.
        \item پس از مبادله، ارتباط امن آغاز می‌شود و داده‌های کاربردی می‌توانند رمزگذاری شوند.
    \end{itemize}
\end{enumerate}

این فریم مراحل نهایی حیاتی handshake \lr{TLS} را تکمیل می‌کند. پس از این مبادله:
\begin{itemize}
    \item کلاینت و سرور به ارتباط رمزگذاری‌شده با استفاده از مجموعه رمزنگاری و کلیدهای جلسه توافق‌شده سوئیچ می‌کنند.
    \item اتصال امن شده و داده‌های رمزگذاری‌شده کاربردی (مانند درخواست‌ها و پاسخ‌های \lr{HTTP}) می‌توانند منتقل شوند.
\end{itemize}
\section*{2. الگوریتم شامیراز}

الگوریتم شامیراز \lr{(Shamir's Secret Sharing)} یک روش ریاضی برای تقسیم یک راز به چند قسمت است به طوری که برای بازیابی راز اصلی نیاز به حداقل تعداد مشخصی از این قسمت‌ها باشد. این الگوریتم بر اساس اصول زیر عمل می‌کند:

\begin{itemize}
    \item یک چندجمله‌ای با درجه $t-1$ به صورت تصادفی ایجاد می‌شود، به طوری که مقدار ثابت چندجمله‌ای برابر با راز اصلی باشد.
    \item با استفاده از این چندجمله‌ای، نقاط مختلفی در دامنه مشخص محاسبه می‌شوند. هر نقطه شامل یک جفت عدد $(x, f(x))$ است که به عنوان یک سهم از راز اصلی در نظر گرفته می‌شود.
    \item برای بازیابی راز اصلی، حداقل $t$ سهم مورد نیاز است. با استفاده از این $t$ نقطه و روش \lr{Lagrange Interpolation} می‌توان چندجمله‌ای اصلی و در نتیجه مقدار راز را بازیابی کرد.
\end{itemize}

این روش دارای ویژگی‌های زیر است:
\begin{itemize}
    \item امنیت: با داشتن کمتر از $t$ سهم، هیچ اطلاعاتی از راز قابل دستیابی نیست.
    \item انعطاف‌پذیری: تعداد سهم‌ها و حداقل تعداد لازم برای بازیابی قابل تنظیم است.
\end{itemize}

\subsection*{توضیح کد شامیراز}

در کد ارائه شده، یک کلاس \lr{Shamir} برای پیاده‌سازی الگوریتم شامیراز ایجاد شده است. توضیح اجزای اصلی کد به صورت زیر است:

\subsection*{ایجاد راز}
متد \lr{create\_secret} یک راز با طول مشخص در بیت تولید می‌کند. برای این کار، از تابع \lr{randbits} کتابخانه \lr{secrets} استفاده می‌شود که یک عدد تصادفی با طول دلخواه در بیت تولید می‌کند. این راز به عنوان مقدار ثابت چندجمله‌ای (ضریب مستقل از \lr{x}) در نظر گرفته می‌شود.

سپس یک چندجمله‌ای تصادفی با درجه t-1 ایجاد می‌شود. این چندجمله‌ای با استفاده از تابع \\\lr{numpy.random.randint} ساخته می‌شود که مقادیر تصادفی بین محدوده مشخصی (در اینجا 50- تا 50) تولید می‌کند. 

در نهایت، مقدار ثابت چندجمله‌ای با راز تولید شده جایگزین می‌شود تا اطمینان حاصل شود که راز بخشی از چندجمله‌ای است. این چندجمله‌ای به کمک \lr{numpy.poly1d} به یک تابع چندجمله‌ای تبدیل می‌شود که می‌تواند برای محاسبه مقادیر در نقاط مختلف استفاده شود.


\subsection*{بازیابی چندجمله‌ای}
متد \lr{polynomial} از روش \lr{Lagrange Interpolation} برای بازسازی چندجمله‌ای اصلی استفاده می‌کند. ورودی‌های $x$ و $y$ که از نقاط تولید شده هستند، در اینجا به صورت تصادفی تولید می‌شوند.

\subsection*{بازیابی راز}
متد \lr{recover\_secret} تعدادی نقاط تصادفی از چندجمله‌ای محاسبه کرده و سپس با استفاده از این نقاط، چندجمله‌ای اصلی را بازسازی می‌کند. مقدار راز اصلی با ارزیابی چندجمله‌ای در \(x = 0\) بازیابی می‌شود.


کد اصلی شامل موارد زیر است:
\begin{itemize}
    \item کاربر تعداد افراد ($n$) و تعداد حداقل افراد لازم ($t$) را وارد می‌کند.
    \item اگر تعداد سهم‌های داده شده کمتر از $t$ باشد، بازیابی ممکن نیست.
    \item در غیر این صورت، چندجمله‌ای بازسازی شده و راز اصلی نمایش داده می‌شود.
\end{itemize}


\subsection*{خروجی}

\includegraphics[width=\linewidth]{images/1.png}

\includegraphics[width=\linewidth]{images/2.png}

\includegraphics[width=\linewidth]{images/3.png}

